\chapter{Einleitung}\label{sec:einleitung}
\section{Überblick}\label{sec:ueberblick}

Der Trend zu vernetzten und intelligenten Fahrzeugen wird die Sicherheit im Straßenverkehr und den Komfort eines Fahrzeugs durch den Datenaustausch mit anderen Systemen in den kommenden Jahren deutlich erhöhen \cite[]{transformation}.

Beispielsweise wurde im neuen VW Golf 8 serienmäßig die Car2X-Technologie eingeführt, die es dem Fahrzeug ermöglicht, mit benachbarten Fahrzeugen oder der Umgebung zu kommunizieren. So können sich die Fahrzeuge gegenseitig vor Gefahren wie einem Stauende warnen \cite[]{Rudschies2020}. In einer angekündigten Version von „Apple CarPlay“ will der Technologiekonzern seine Inhalte nicht nur auf dem Infotainmentsystem, sondern auch auf Instrumentenanzeigen wie dem Tacho darstellen \cite[]{Nellis2022}.

Die daraus resultierende Zunahme der Komplexität von Steuergeräteaufgaben und der externen Kommunikation eines Fahrzeugs eröffnet jedoch neue Angriffsvektoren, die mit besonderen Anforderungen an die Robustheit der Geräte kompensiert werden müssen. 

\begin{figure}
	\centering
	\includegraphics[width=\textwidth]{images/FahrzeugNetz.pdf}
	\caption{Vereinfachte Darstellung der Netzwerkarchitektur eines Fahrzeugs}
	\label{fig:fahrzeugnetze}
\end{figure}

Um der wachsenden Gefahr von Hackerangriffen zu begegnen, versuchen Automobilhersteller, Schwachstellen so früh wie möglich im Entwicklungsprozess eines Fahrzeugs zu finden. Eine bewährte Methode zum Aufspüren von Schwachstellen ist dabei das Pentesting \cite[]{Ebert2021}. Für die IP-basierte Kommunikation gibt es dazu eine breite Palette von Softwaretools auf dem Markt, die jedoch hauptsächlich für klassische IT-Systeme konzipiert wurden. Auch ein Teil der Kommunikation in Fahrzeugen ist IP-basiert, auf die allerdings die meisten Werkzeuge nicht ausgelegt sind. 

IP basierte Kommunikation findet beispielsweise zwischen einer Ladesäule und einem Fahrzeug, bei Automotive Ethernet basierten Netzen oder mit einem Diagnosegeräte über OBD statt (siehe Abbildung \ref{fig:fahrzeugnetze}).

\section{Stand der Technik}\label{sec:ziel}

In dem Buch „The Car Hacker’s Handbook“ beschreibt \cite{carhacker} in Kapitel 5, wie ein CAN-Bus mit den Anwendungen Wireshark und Candump analysiert werden kann, um potenzielle Sicherheitslücken zu entdecken. Zum Testen von Komponenten schlägt der Autor in Kapitel 7 des Buchs einen Testaufbau vor, bei dem Sensoren eines Steuergeräts mit Mikrocontrollern wie einem Arduino simuliert werden, um es auf diese Weise isoliert untersuchen zu können. In Kapitel 11 geht der Autor darauf ein, wie man gefundene Schwachstellen mit dem Framework Metasploit für einen Angriff nutzen kann. 

Das Exploit-Framework Metasploit bietet mit der Automotive Extension unter dem Pfad \path{post/hardware/automotive/} neun Module, die allesamt auf den CAN-Bus eines Fahrzeugs abzielen. 

\cite{automotivepentestingos} stellte in einem Vortrag vor der „British Computer Society Open Source Specialists“ seine Analyse von „Car-Hacking“-Tools vor und stellte darin fest, dass lediglich für sehr spezifische Anwendungsfälle Tools existieren. Aufgrund dieses Mangels an Tools erweiterte er die Anwendung Scapy (siehe Kapitel \ref{subsec:scapy}) um Features für die Analyse von Automotive-Protokollen und schuf so ein leistungsfähiges Werkzeug für das Pentesting im Automotive-Bereich. Für die Simulation von Steuergeräten demonstriert er in dem Vortrag einen automatisierten Prüfstand, der unter anderem aufgezeichneten Datenverkehr eines Testfahrzeugs als Grundlage für die Simulation von Steuergeräten verwendete.

Zusammenfassend lässt sich feststellen, dass Pentesting im Automobilbereich mit einem enormen Aufwand verbunden ist, da der Zugriff auf und die Simulation von Steuergeräte mit den derzeit existierenden Anwendungen nur schwer zu realisieren ist. Bis auf Scapy sind alle Tools in diesem Sektor lediglich auf den CAN-Bus fokussiert. Nur Scapy bietet ein aktiv entwickeltes Tool für das Pentesting einer Vielzahl von Automobilprotokollen, das allerdings nur Komponenten für die Entwicklung und keine konkreten Exploits enthält. 

Eine mit Metasploit vergleichbare Lösung, die eine Vielzahl von vorgefertigten Modulen zum Scannen und Ausnutzen von Exploits enthält, existiert derzeit nicht für den Automobilsektor. 

\section{Ziel der Arbeit}\label{sec:ziel}

Ziel dieser Arbeit ist es, zu untersuchen, wie Pentests in einem Automotive-Testsysteme mit etablierten Tools wie dem Exploit-Framework Metasploit ermöglicht werden können. In Automotive-Testsystemen wie CANoe können komplette Kommunikationsnetzwerke eines Fahrzeugs simuliert und getestet werden, wodurch sie eine perfekte Umgebung für Pentests schaffen. Auf diese Weise können komplexe Szenarien simuliert werden, in denen ein Fahrzeug besonders anfällig für Angriffe sein könnte. Zudem kann durch einen rein softwarebasierten Zugang viel manuelle Arbeit für den physikalischen Zugriff zu den Netzen vermieden werden. 

\section{Vorgehen}\label{sec:vorgehen}

In Kapitel \ref{sec:anforderungen} werden zunächst Anforderungen an die Anwendung aufgezeigt. Dazu wurde untersucht, was laut Bundesamt für Sicherheit in der Informationstechnik (BSI) für einen Pentest relevant ist. Für den Bezug zum Automobilsektor wurden Normen und Vorschriften im Automobilbereich ermittelt, die Pentests als Validierungsmaßnahmen vorsehen. Organisatorisch wurde untersucht, in welchem Teil des Entwicklungsprozesses von Steuergerätesoftware ein Pentest sinnvoll ist und ob für bestimmte Aufgaben bereits Softwarewerkzeuge existieren. Aus den gefundenen Anforderungen wurde eine Architektur entwickelt, die diese erfüllt. Diese diente als Grundlage für die Implementierung im Kapitel \ref{sec:implementierung}, in dem verschiedene Implementierungsmöglichkeiten verglichen und implementiert wurden. In der Evaluierungsphase in Kapitel \ref{sec:evaluation} wurde die Implementierung kontinuierlich evaluiert und entsprechend angepasst, um einen Pentest der ISO 15118-Ladekommunikation durchzuführen. Für den Pentest der Ladekommunikation wurden bekannte Schwachstellen untersucht und Angriffe implementiert, die mit der geschaffenen Anwendung ausgeführt werden können. 