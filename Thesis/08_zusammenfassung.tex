\chapter{Fazit und Ausblick}\label{sec:zusammenfassung}

\section{Überblick}\label{sec:ueberblickfazit}

In diesem Projekt wurde eine Strategie zur Anbindung eines Exploit-Frameworks erforscht. Dazu wurde ein Prototyp einer Anbindung implementiert und anhand eines Pentests der ISO 15118 Ladekommunikation kontinuierlich evaluiert und erweitert. 

Durch die Zweckentfremdung der tapserver-Funktionalität von CANoe konnte eine zuverlässige Verbindung zwischen Linux-Anwendungen und CANoe-Netzwerken hergestellt werden. Docker erwies sich als die Umgebung der Wahl für die Ausführung der Linux-Anwendungen, vordergründig wegen seines hohen Automatisierungsgrades. Dank dieses hohen Grades konnte die Anbindung von Metasploit so implementiert werden, als wäre es nur eine weitere Komponente von CANoe. Durch die Verwendung von Docker-Netzwerktreibern konnte eine komplexe Netzwerkkonfiguration geplant und konfiguriert werden. Metasploit wurde um Funktionalitäten wie den Export der Ergebnisse einer Modulausführung in eine Redis-Datenbank erweitert, wodurch die Auswertung der ausgeführten Module ermöglicht wurde. Die Benutzeroberfläche von PentestStudio ermöglicht eine umfassende Fernsteuerung von Metasploit, ohne die Oberfläche von CANoe jemals verlassen zu müssen. Da das Ausführen von Modulen in PentestStudio komplett geloggt wird, können umfangreiche Reports über durchgeführte Untersuchungen erstellt werden, die als Grundlage weiterer Anwendungen verwendet werden können. Der Code-Generator von PentestStudio ermöglicht eine schnelle Integration von Modulen in die vTESTstudio-Testumgebung, was die Dokumentation von Schwachstellen und die Erstellung von Regressionstests erheblich erleichtert. 

Die in Metasploit enthaltenen Module sind für Automotive Pentesting kaum verwendbar und eignen sich hauptsächlich für bekannte Schwachstellen von Serveranwendungen, die in Automobilnetzen für gewöhnlich nicht vorkommen. Allgemeine Module wie Port-Scanner erwiesen sich im Vergleich zu bekannten Open-Source-Tools stets als funktional unterlegen, weshalb Kommandozeilen Tools wie nmap oder thc-ipv6 als Metasploit Modul verpackt wurden. 

Die selbst entwickelten Metasploit-Module ermöglichen einfache Angriffe auf die Ladekommunikation des ISO 15118 Standards. Durch die konkrete Nutzung der prototypisch implementierten Anwendung für das Pentesting konnte während der Implementierung kontinuierlich Feedback gesammelt werden. 

\section{Erweiterungsmöglichkeiten}\label{sec:erweiterungsmoeglichkeiten}

Zum einen wäre es möglich, die Installation von Docker Desktop oder Podman in das Installationsprogramm von CANoe einzubinden, was dem Endanwender die zusätzliche Installation der Anwendung ersparen würde. Derzeit erfolgt die Einrichtung des Docker-Containers über eine Docker Compose-Datei. Dieser Vorgang könnte manuell in einem Dialog der Anwendung oder automatisiert aus den Einstellungen eines CANoe Projekts erfolgen.

Mit der Weiterentwicklung der Tapserver-Funktionalität könnte es in Zukunft möglich sein, CAN, LIN, etc. über die Anwendung mit dieser oder einer ähnlichen Funktion zu testen.

Die Zuweisung einer Payload für einen Metasploit-Exploit ist in der aktuellen Version der Anwendung umständlich, da zwei Module nacheinander geladen werden müssen. In Zukunft könnte die Konfiguration in der Benutzeroberfläche entsprechend visualisiert werden. Das Auffinden eines Metasploit-Moduls erfordert derzeit die Verwendung einer Webanwendung wie \url{www.exploit-db.com} und \url{www.rapid7.com/db/} oder die Erstellung einer Ordnerstruktur mit gespeicherten PentestStudio-Projekten. Das Archiv der installierten Exploits könnte in einer zukünftigen Version mit einem Exploit-Browser visualisiert werden. 

%Um häufig verwendete Exploit-Module schneller laden zu können, wäre unter dem Menüpunkt \path{File} ein Eintrag \path{Recent Modules} sehr hilfreich.

Das Tool verlässt sich auf Metasploit als zentrale Ausführungsumgebung. Um andere Module nutzen zu können, wurde Metasploit um Funktionalitäten wie den \path{CLI-Runner} erweitert. In einer zukünftigen Version könnte PentestStudio möglicherweise eine Abstraktionsschicht auf einem Linux-Rechner steuern, die wiederum Metasploit steuern könnte. Für Anwendungsfälle, die nicht auf Metasploit angewiesen sind, würde dies die Entwicklung von Tools vereinfachen, da es nicht notwendig wäre, das Schema eines Metasploit-Moduls zwanghaft zu befolgen.

PentestStudio ist derzeit nur ein Panel in einem CANoe Projekt. Es wäre sinnvoll, die Anwendung als Teil von CANoe zu integrieren. Der Report der ausgeführten Module ist derzeit nur ein JSON und sollte mehr wie die Reports von vTESTstudio aussehen. 

Die entwickelten Metasploit-Module wurden größtenteils für den spezifischen Testfall dieser Arbeit entwickelt und müssten für die weitere Verwendung generischer gestaltet werden.

\section{Fazit}\label{sec:fazit}

Im Kontext von Fahrzeugen werden unter „Safety“ Merkmale verstanden, die verhindern sollen, dass ein Fahrzeug Schäden an sich selbst, der Umwelt oder am Fahrer verursacht. Bekannte Merkmale dieser Kategorie sind unter anderem der Airbag, das ABS oder das ESP. Unter „Security“ werden dagegen Merkmale verstanden, die dem Schutz des Fahrzeugs dienen sollen. Diese beiden Merkmale können jedoch nicht unabhängig voneinander betrachtet werden, da eine mangelhafte „Security“ auch die „Safety“ eines Fahrzeugs beeinträchtigen kann. Wird ein sicherheitskritisches Steuergerät durch einen Angriff kompromittiert, kann es im schlimmsten Fall seine „Safety“ Aufgaben nicht mehr erfüllen. Aus diesem Grund sollten „Security“ Merkmale immer als indirekte „Safety“ Merkmale betrachtet werden. Mit dieser Arbeit wurde der Grundstein für den verstärkten Einsatz von Pentesting als Validierungsmethode der „Security“ von Fahrzeugsoftware gelegt, wodurch auch zukünftig das hohe „Safety“-Niveau, das von modernen Fahrzeugen erwartet wird, gewährleistet werden kann.