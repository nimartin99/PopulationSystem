% durch Austauschen dieser Zeilen kann die Sprache des Templates geändert werden
\PassOptionsToPackage{main=ngerman}{babel}
%\PassOptionsToPackage{main=english}{babel

% durch Austauschen dieser Zeilen kann zwischen Abschlussarbeit und Seminararbeit gewechselt werden
\documentclass[thesis, numbers=noenddot]{mi-document}
%\documentclass[seminar]{mi-document}

\DeclareTOCStyleEntry[
  level=\chaptertocdepth,
  indent=0pt,
  numwidth=2.3em,
  dynnumwidth,
  linefill=\hfill,
  entryformat=\appendixtocformat,
  entrynumberformat=\appendixtocnumberformat,
  pagenumberformat=\appendixtocpagenumberformat
]{tocline}{appendixchapter}
\newcommand*\appendixtocformat[1]{{\usekomafont{chapterentry}#1}}
\newcommand*\appendixtocnumberformat[1]{{\def\autodot{:}\appendixname\ #1}}
\newcommand*\appendixtocpagenumberformat[1]
  {{\usekomafont{chapterentry}\usekomafont{chapterentrypagenumber}#1}}

\usepackage{xpatch}
\xapptocmd\appendix
  {%
    \renewcommand*{\chapterformat}{%
      \mbox{\appendixname{\nobreakspace}\thechapter:%
        \IfUsePrefixLine{}{\enskip}}%
    }%
    \renewcommand*{\chaptermarkformat}{\appendixname\ \thechapter:\enskip}%
    \xpatchcmd{\addchaptertocentry}
      {\addtocentrydefault{chapter}{#1}{#2}}
      {\addtocentrydefault{appendixchapter}{#1}{#2}}%
      {}{\PatchFailed}%
  }{}{\PatchFailed}

\bachelor % im Falle einer Masterarbeit \master

% Variablen, die für das Deckblatt und Metadaten verwendet werden
\title{Anbindung eines Exploit-Frameworks an ein Automotive-Testsystem}
\author{Leonhard Bosch}
\semester{SS22}
\course{Bachelorarbeit}
\module{[z.B. MEI-M 04 (B.A.)]}
\dozent{Prof. Dr.-Ing. Andreas Mayer}
\studid{204002}
\studSemester{7. Semester B.S. Software Engineering}
\phone{0941/133742666} % Optional
\studSubject{Software Engineering}
\firstReviewer{Prof. Dr.-Ing. Andreas Mayer}
\secondReviewer{Dipl.-Ing. Holger Heinemann}
\advisor{Holger Heinemann}
\mail{leonhard.bosch@icloud.com}
\studMail{lbosch@stud.hs-heilbronn.de}
\dateHandedIn{30.06.2022}
\keywords{automotive;cyber security;pentest;software testing}

\bibliographystyle{apacite}

% Falls Sie die Abkürzung zum Einbinden von Grafiken benutzen möchten. Erläuterung fnden Sie im Abschnitt zu Abbildungen.
\newcommand*{\image}[2]{
	\begin{figure}
	\centering
	\includegraphics[width=0.5\textwidth]{{images/#1}}
	\caption{#2}
	\label{fig:#1}
	\end{figure}
}

\begin{document}

% Die Nummerierung beginnt mit der Titelseite (= Seite 1), soll aber erst ab der ersten Inhaltsseite (Einleitung) angezeigt werden.
\pagestyle{empty}

% Das Deckblatt erstellen
\maketitle

%\printindex

\pagestyle{headings} % Seitennummern und Kapitelbezeichnungen anzeigen

\input{00_abstract}

\tableofcontents % Optional
\listoffigures % Optional
%\listoftables % Optional
%\lstlistoflistings % Optional


\clearpage
\onehalfspacing

\clearpage

% hier beginnt der eigentliche Inhalt der Arbeit

\addchap{Vorwort}\label{sec:vorwort}

Blub
\chapter{Einleitung}\label{sec:einleitung}
\section{Überblick}\label{sec:ueberblick}

Der Trend zu vernetzten und intelligenten Fahrzeugen wird die Sicherheit im Straßenverkehr und den Komfort eines Fahrzeugs durch den Datenaustausch mit anderen Systemen in den kommenden Jahren deutlich erhöhen \cite[]{transformation}.

Beispielsweise wurde im neuen VW Golf 8 serienmäßig die Car2X-Technologie eingeführt, die es dem Fahrzeug ermöglicht, mit benachbarten Fahrzeugen oder der Umgebung zu kommunizieren. So können sich die Fahrzeuge gegenseitig vor Gefahren wie einem Stauende warnen \cite[]{Rudschies2020}. In einer angekündigten Version von „Apple CarPlay“ will der Technologiekonzern seine Inhalte nicht nur auf dem Infotainmentsystem, sondern auch auf Instrumentenanzeigen wie dem Tacho darstellen \cite[]{Nellis2022}.

Die daraus resultierende Zunahme der Komplexität von Steuergeräteaufgaben und der externen Kommunikation eines Fahrzeugs eröffnet jedoch neue Angriffsvektoren, die mit besonderen Anforderungen an die Robustheit der Geräte kompensiert werden müssen. 

\begin{figure}
	\centering
	\includegraphics[width=\textwidth]{images/FahrzeugNetz.pdf}
	\caption{Vereinfachte Darstellung der Netzwerkarchitektur eines Fahrzeugs}
	\label{fig:fahrzeugnetze}
\end{figure}

Um der wachsenden Gefahr von Hackerangriffen zu begegnen, versuchen Automobilhersteller, Schwachstellen so früh wie möglich im Entwicklungsprozess eines Fahrzeugs zu finden. Eine bewährte Methode zum Aufspüren von Schwachstellen ist dabei das Pentesting \cite[]{Ebert2021}. Für die IP-basierte Kommunikation gibt es dazu eine breite Palette von Softwaretools auf dem Markt, die jedoch hauptsächlich für klassische IT-Systeme konzipiert wurden. Auch ein Teil der Kommunikation in Fahrzeugen ist IP-basiert, auf die allerdings die meisten Werkzeuge nicht ausgelegt sind. 

IP basierte Kommunikation findet beispielsweise zwischen einer Ladesäule und einem Fahrzeug, bei Automotive Ethernet basierten Netzen oder mit einem Diagnosegeräte über OBD statt (siehe Abbildung \ref{fig:fahrzeugnetze}).

\section{Stand der Technik}\label{sec:ziel}

In dem Buch „The Car Hacker’s Handbook“ beschreibt \cite{carhacker} in Kapitel 5, wie ein CAN-Bus mit den Anwendungen Wireshark und Candump analysiert werden kann, um potenzielle Sicherheitslücken zu entdecken. Zum Testen von Komponenten schlägt der Autor in Kapitel 7 des Buchs einen Testaufbau vor, bei dem Sensoren eines Steuergeräts mit Mikrocontrollern wie einem Arduino simuliert werden, um es auf diese Weise isoliert untersuchen zu können. In Kapitel 11 geht der Autor darauf ein, wie man gefundene Schwachstellen mit dem Framework Metasploit für einen Angriff nutzen kann. 

Das Exploit-Framework Metasploit bietet mit der Automotive Extension unter dem Pfad \path{post/hardware/automotive/} neun Module, die allesamt auf den CAN-Bus eines Fahrzeugs abzielen. 

\cite{automotivepentestingos} stellte in einem Vortrag vor der „British Computer Society Open Source Specialists“ seine Analyse von „Car-Hacking“-Tools vor und stellte darin fest, dass lediglich für sehr spezifische Anwendungsfälle Tools existieren. Aufgrund dieses Mangels an Tools erweiterte er die Anwendung Scapy (siehe Kapitel \ref{subsec:scapy}) um Features für die Analyse von Automotive-Protokollen und schuf so ein leistungsfähiges Werkzeug für das Pentesting im Automotive-Bereich. Für die Simulation von Steuergeräten demonstriert er in dem Vortrag einen automatisierten Prüfstand, der unter anderem aufgezeichneten Datenverkehr eines Testfahrzeugs als Grundlage für die Simulation von Steuergeräten verwendete.

Zusammenfassend lässt sich feststellen, dass Pentesting im Automobilbereich mit einem enormen Aufwand verbunden ist, da der Zugriff auf und die Simulation von Steuergeräte mit den derzeit existierenden Anwendungen nur schwer zu realisieren ist. Bis auf Scapy sind alle Tools in diesem Sektor lediglich auf den CAN-Bus fokussiert. Nur Scapy bietet ein aktiv entwickeltes Tool für das Pentesting einer Vielzahl von Automobilprotokollen, das allerdings nur Komponenten für die Entwicklung und keine konkreten Exploits enthält. 

Eine mit Metasploit vergleichbare Lösung, die eine Vielzahl von vorgefertigten Modulen zum Scannen und Ausnutzen von Exploits enthält, existiert derzeit nicht für den Automobilsektor. 

\section{Ziel der Arbeit}\label{sec:ziel}

Ziel dieser Arbeit ist es, zu untersuchen, wie Pentests in einem Automotive-Testsysteme mit etablierten Tools wie dem Exploit-Framework Metasploit ermöglicht werden können. In Automotive-Testsystemen wie CANoe können komplette Kommunikationsnetzwerke eines Fahrzeugs simuliert und getestet werden, wodurch sie eine perfekte Umgebung für Pentests schaffen. Auf diese Weise können komplexe Szenarien simuliert werden, in denen ein Fahrzeug besonders anfällig für Angriffe sein könnte. Zudem kann durch einen rein softwarebasierten Zugang viel manuelle Arbeit für den physikalischen Zugriff zu den Netzen vermieden werden. 

\section{Vorgehen}\label{sec:vorgehen}

In Kapitel \ref{sec:anforderungen} werden zunächst Anforderungen an die Anwendung aufgezeigt. Dazu wurde untersucht, was laut Bundesamt für Sicherheit in der Informationstechnik (BSI) für einen Pentest relevant ist. Für den Bezug zum Automobilsektor wurden Normen und Vorschriften im Automobilbereich ermittelt, die Pentests als Validierungsmaßnahmen vorsehen. Organisatorisch wurde untersucht, in welchem Teil des Entwicklungsprozesses von Steuergerätesoftware ein Pentest sinnvoll ist und ob für bestimmte Aufgaben bereits Softwarewerkzeuge existieren. Aus den gefundenen Anforderungen wurde eine Architektur entwickelt, die diese erfüllt. Diese diente als Grundlage für die Implementierung im Kapitel \ref{sec:implementierung}, in dem verschiedene Implementierungsmöglichkeiten verglichen und implementiert wurden. In der Evaluierungsphase in Kapitel \ref{sec:evaluation} wurde die Implementierung kontinuierlich evaluiert und entsprechend angepasst, um einen Pentest der ISO 15118-Ladekommunikation durchzuführen. Für den Pentest der Ladekommunikation wurden bekannte Schwachstellen untersucht und Angriffe implementiert, die mit der geschaffenen Anwendung ausgeführt werden können. 
\chapter{Grundlagen}\label{sec:grundlagen}

\section{Steuergeräte}\label{subsec:steuergeraete}

Im Jahr 1967 führte VW erstmals die elektronische Motorsteuerung D-Jetronic von Bosch in einem Serienfahrzeug ein. Dieses System ermöglichte es erstmals, das Verhältnis des Kraftstoff-Luft-Gemischs elektronisch entsprechend dem aktuellen Betriebszustand des Motors anzupassen und so den Kraftstoffverbrauch zu reduzieren \cite[S. 13]{schaeffer2013}. Das damals noch als futuristisch gesehene Konzept ist heute aus der Motorensteuerung nicht mehr wegzudenken und kennzeichnet den Beginn des Siegeszugs elektronischer Steuergeräte.

Laut James Priyatham enthält ein modernes Luxusfahrzeug über 150 Steuergeräte \cite[]{numberEcus}. Diese umfassen die bereits erwähnten Motorsteuergeräte, aber auch Fahrsicherheitssysteme wie das ABS oder den Airbag. Verstärkt finden sich immer mehr Steuergeräte, die einzig allein dem Fahrkomfort wie Lenkassistenten oder einem Infotainment System dienen \cite[S. 1]{Wolf2018}. 

% Mehr Historie über entwicklung, 
% Von CAN bis Ethernet

\section{AUTOSAR}\label{subsec:autosarclassic}

Um die wachsende Komplexität von Steuergeräten in modernen Fahrzeugen zu reduzieren, gründeten sieben deutsche Kfz-Hersteller und Zulieferer 2003 die AUTOSAR-Entwicklungspartnerschaft, in der der De-Facto Industriestandard AUTOSAR Classic erarbeitet wurde. Dieser immer wieder aktualisierte Standard definiert in \cite{AUTOSAR2020} die folgenden Projektziele:
\begin{enumerate}
\item Übertragbarkeit von Software
\item Skalierbarkeit für unterschiedliche Fahrzeuge und Plattformvarianten
\item Unterstützung einer Vielzahl von Funktionsdomänen
\item Definition einer offenen Architektur
\item Unterstützung der Entwicklung von zuverlässigen Systemen
\item Nachhaltige Nutzung natürlicher Ressourcen
\item Standardisierung von Basissoftwarefunktionalität von Steuergeräten
\item Unterstützung von relevanten internationalen Automobilstandards und etablierten technischen Lösungen
\end{enumerate}

\begin{figure}
	\centering
	\includegraphics[width=\textwidth]{images/autosarclassic.pdf}
	\caption{AUTOSAR Classic Architektur \\ Quelle: Eigene Darstellung nach \cite{autosarlayers}}
	\label{fig:autosarlayers}
\end{figure}

Um dies zu erreichen, definiert AUTOSAR neben anderen Aspekten wie der Entwicklungsmethodik eine Referenzarchitektur für Steuergeräte-Software, die die eigentliche Hardware eines Steuergeräts über die folgenden drei Schichten für Anwendungen abstrahiert:

\begin{itemize}
\item \textbf{Application Layer} Software, die vom Hersteller für den Verwendungszweck des Steuergeräts entwickelt wird
\item \textbf{Runtime Enviroment} Definiert die Kommunikation zwischen der Application Layer und der Basissoftware
\item \textbf{Basic Software} Softwarekomponente, die grundlegende Standarddienste des Steuergeräts bereitstellen
\end{itemize}

Die Basic Software wird wiederum in die folgenden drei Schichten eingeteilt:
\begin{itemize}
\item \textbf{Microcontroller Abstraction Layer} Treiber für die spezifische Hardware
\item \textbf{ECU Abstraction Layer} Hardwareunabhängige Schnittstelle für Hardwarearten
\item \textbf{Services Layer} Schnittstellen für konkrete Services wie kryptografische Funktionen
\item \textbf{Complex Drivers} Services, die nicht auf den vorherigen Schichten aufbauen, da sie z. B. zeitkritisch sind
\end{itemize}

\cite[]{autosarlayers}

Ein potenzielles Sicherheitsrisiko sehen \citet{attackingAutosar} in der Abstraktionsschicht von Mikrocontrollern. Im Gegensatz zu anderer Basissoftware werden diese Komponenten oft von den Mikrocontroller-Herstellern selbst implementiert. Wenn diese den AUTOSAR-Standard nicht ordnungsgemäß implementierten, kann es sein, dass bei der Integration mit der sonstigen Software des Steuergeräts Sicherheitslücken in das System eingeführt werden.

Zudem stellen die für den AUTOSAR-Standard implementierten Kommunikationsstacks ein Sicherheitsrisiko dar, da sie aufgrund ihres Nischendaseins nicht so gut getestet wurden wie etablierte Kommunikationsstacks wie der des Linux-Kernels. 

\subsection{AUTOSAR Kommunikationsstacks}\label{subsec:kommunikationsstacks}

Der AUTOSAR Standard benennt die folgenden Netzwerkprotokolle als Teil des Kommunikationsstacks:

\subsubsection{CAN}\label{subsec:can}

Das Controller Area Network (CAN) ist ein von der Robert Bosch GmbH erfundener Kommunikationsbus, der 1986 auf dem Kongress der Society of Automotive Engineers (SAE) in Detroit als Alternative zur damals üblichen Punkt-zu-Punkt-Verkabelung in Fahrzeugen vorgestellt wurde. Nachrichten in CAN werden auf dem Bus an alle Steuergeräte übertragen, welche anhand des \path{Message Identifiers} entscheiden, ob die gesendete Nachricht von ihnen weiterverarbeitet wird oder nicht.

CAN verwendet ein CSMA/CA-Verfahren (Carrier-sense multiple access with collision avoidance) für den Buszugriff. Dabei beobachtet jedes Steuergerät den Bus für einen bestimmten Zeitraum ohne Aktivität, bevor es versucht, eine Nachricht zu senden. Jeder Teilnehmer des Busses hat die gleiche Chance, eine Nachricht zu senden, was potenziell zu Nachrichtenkollisionen führen kann. Beim Senden jedes Bits des \path{Message Identifiers} am Anfang einer Nachricht überprüft der Absender, ob die gesendeten Daten mit den empfangenen Daten übereinstimmen. Stimmen die Bits nicht überein, weiß der Absender, dass zur gleichen Zeit eine Nachricht mit höherer Priorität auf dem Bus übertragen wurde und bricht den Sendevorgang ab. Ein erneuter Sendeversuch wird erst wieder unternommen, wenn eine bestimmte Zeitspanne ohne Aktivität verstrichen ist \cite[S. 57]{bussyteme}.

\subsubsection{LIN}\label{subsec:LIN}

Bei LIN (Local Interconnect Network) handelt es sich um ein Ende der 1990er-Jahre erschienenes Bussystem, das kostengünstiger als CAN ist, allerdings einen geringeren Datendurchsatz aufweist. Anders als CAN, handelt es sich bei LIN um einen Master/Slave Bus. Sämtliche Nachrichten gehen von einem einzigen Master aus, dem die Slaves antworten, wodurch Nachrichtenkollisionen verhindert werden \cite[S. 79]{bussyteme}.

\subsubsection{FlexRay}\label{subsec:Flex}

Im Jahr 1999 gründeten BMW, Daimler, Motorola und Philips das FlexRay-Konsortium, um ein fehlertolerantes Kommunikationssystem mit Übertragungsraten von bis zu 10 Mbit/s zu entwickeln.

Neben den bekannten Bustopologien unterstützt FlexRay auch Sterntopologien, bei denen ein zentraler Knoten die Weiterleitung der Nachrichten übernimmt. 

Das Protokoll arbeitet zyklisch und teilt die Datenübertragung in zwei Teile, einen statischen und einen dynamischen Teil. Im statischen Teil wird jedem Knoten des Busses eine kleine Zeitspanne eingeräumt, in der er ungestört Nachrichten senden kann. Im dynamischen Teil kann wie bei CAN jeder Teilnehmer nach Belieben Nachrichten senden. Ferner enthält der Zyklus ein kleines Zeitfenster, welches die Synchronisation der Teilnehmer erlaubt \cite[S. 96]{bussyteme}.

\subsubsection{Automotive Ethernet}\label{subsec:ethernet}

Bei Automotive Ethernet handelt es sich um eine Sonderform von Ethernet, die auf die Bedürfnisse der Automobilindustrie zugeschnitten wurde. Ethernet wurde in den 70er-Jahren vom Xerox Palo Alto Research Center  für die Vernetzung von Computer Workstations entwickelt, 1980 in der ersten Version des Standards IEEE 802 standardisiert und über die Jahre fortlaufend weiterentwickelt \cite[S. xxxii]{automotiveethernet}.

Obwohl Ethernet in den Anfangsjahren eine Bustopologie unterstütze, handelt es sich bei Automotive Ethernet hauptsächlich um eine Sterntopologie.

Für die Verwendung in der Automobilindustrie wurde Ethernet so weiter entwickelt, dass es den besonderen Anforderungen der Branche gerecht wurde. Zu den besonderen Anforderungen zählen unter anderem die Einhaltung der Grenzwerte für abgestrahlte elektromagnetische Störgrößen \cite[S. 109]{automotiveethernet}, wie aber auch eine große Resilienz für Temperaturen, welche im Fahrzeuginneren von -40 \textdegree{}C bis +105 \textdegree{}C reichen können \cite[S. 320]{automotiveethernet}. IEEE Standards, welche diese besonderen Anforderungen erfüllen, sind 100BASE-T1, 1000BASE-T1 so wie 100BASE-TX \cite[S. 217]{automotiveethernet}

Im Open Systems Interconection (OSI)- Modell lässt sich Automotive Ethernet in der untersten Schicht als Bitübertragungsschicht einordnen. Neben den verwendeten Ethernet Standard meint man aber mit Automotive Ethernet auch meist die folgenden Protokolle, welche über der zweiten Schicht einzuordnen sind.

\begin{table}[H]
\resizebox{\textwidth}{!}{%
\begin{tabular}{l|llll}
\textbf{Netzwerkprotokoll} & \textbf{CAN (ISO 11898)} & \textbf{LIN} & \textbf{FlexRay} & \begin{tabular}[c]{@{}l@{}}\textbf{Ethernet }\\\textbf{(100BASE-T1)}\end{tabular} \\ 
\hline
\textbf{Topologie} & BUS & BUS & BUS, Stern & Stern \\
\textbf{Nutzdatenraten} &  1 mbit/s &  20 kbit/s &  10 mbit/s &  100 mbit/s \\
\textbf{Kabel Typ} & zweiadrig & einadrig & \begin{tabular}[c]{@{}l@{}}Glasfaser \\oder zweiadrig\end{tabular} & zweiadrig \\
\textbf{Anwendung} & \begin{tabular}[c]{@{}l@{}}Motorsteuerung, \\Scheinwerfer, \\Klimaanlage\end{tabular} & \begin{tabular}[c]{@{}l@{}}Fensterheber, \\Temperatursensoren, \\Klimaanlagen-\\ventilatoren\end{tabular} & \begin{tabular}[c]{@{}l@{}}Lenkwinkelsensor, \\Getriebesteuerung, \\Notbremsassistent\end{tabular} & \begin{tabular}[c]{@{}l@{}}Infotainment, \\Kameras, \\Radare\end{tabular}
\end{tabular}
}
  \caption{Überblick Netzwerkprotokolle}\label{tab:netzwerkprotokole}
\end{table}

\subsection{Automotive Ethernet Protokolle}\label{subsec:ethernetprotokolle}

Während bei den Kommunikationsstacks, die primär für den Automobilsektor entwickelt wurden, häufig die Application Layer direkt auf dem Data Link aufbaut, enthält Automotive Ethernet eine breite Auswahl an Protokollen, die sich auf den unterschiedlichsten Layern des OSI Modells ansiedeln.

\subsubsection{SOME/IP}\label{subsec:someip}

Bei SOME/IP (Scalable service-Oriented Middleware over IP) handelt es sich um eine Middleware, welche eine serviceorientierte Übertragung von Informationen ermöglicht. Anders als bei der signalbasierten Kommunikation, werden hier lediglich Informationen versendet, wenn ein Empfänger mitgeteilt hat, dass er diese benötigt.

Um dies zu ermöglichen, stellt SOME/IP die folgenden Servicekonzepte zur Verfügung:
\begin{itemize}
\item \textbf{Request/Response} Der Client sendet eine Nachricht an den Server und erhält eine Antwort vom Server
\item \textbf{Fire and Forget} Der Client sendet  eine Nachricht an den Server, erwartet jedoch keine Antwort vom Server
\item \textbf{Events} Der Client abonniert auf dem Server eine Information, die der Server bei bestimmten Ereignissen oder regelmäßig an den Client sendet
\item \textbf{Fields} Der Client kann ein Feld lesen oder je nach Konfiguration auch schreiben und erhält zyklisch oder bei einer Veränderung den aktuellen Wert des Felds
\end{itemize} 
\cite[S. 289]{automotiveethernet}

Für die Erkennung, welche Services in einem Netzwerk angeboten werden, enthält SOME/IP zusätzlich einen Service Discorvery Mechanismus. Dies ist z. B. beim Start des Fahrzeugs sinnvoll, da Steuergeräte unterschiedlich lange brauchen, um zu starten, und so das Fahrzeug bereits frühzeitig mit einem limitierten Funktionsumfang parat steht. Es kann auch sein, dass ein bestimmtest Steuergerät wegen einer zu niedrigen Versorgungsspannung, eines Defekts oder einfach, weil es in der Fahrzugkonfiguration des Kunden nicht enthalten ist, nicht kontaktierbar ist und deshalb andere Steuergeräte darauf entsprechend reagieren müssen \cite[S. 291]{automotiveethernet}.

%\subsubsection{Protocol Data Units}\label{subsec:pdu}
%\cite{neuekommunikation}

\subsubsection{DoIP}\label{subsec:doip}

Bei DoIP handelt es sich um ein Diagnoseprotokoll, das direkt mit den Steuergeräten im Ethernet Netzwerk des Fahrzeugs über den Diagnoseport kommuniziert.
Dafür benötigt das Diagnosegerät zunächst eine IP, die es entweder über einen DHCP Server im Netz des Fahrzeuges erhält oder im Fall, dass kein DHCP Server antwortet, zufällig generiert. 
Im nächsten Schritt sendet das Diagnosegerät einen Vehicle Announcement Broadcast, woraufhin sich alle Steuergeräte beim Diagnosegerät melden. An das passende Steuergerät versendet das Diagnosegerät eine „routing activation request“, woraufhin die Diagnosekommunikation startet\cite[]{doip}.

\section{ISO 15118 Protokole}\label{subsec:fahrzeugkommunikation}

\begin{figure}[H]
	\centering
	\includegraphics[width=\textwidth]{images/OSI-15118.pdf}
	\caption{ISO 15118 im OSI-Modell}
	\label{fig:iso15118}
\end{figure}

Nicht nur über Automotive Ethernet findet in einem Fahrzeug IP basierte Kommunikation statt, sondern auch bei der Kommunikation zwischen einer Ladesäule und einem Fahrzeug. Wie diese Kommunikation auszusehen hat, ist in ISO 15118 beschrieben.  

\subsubsection{HomePlug GreenPhy}\label{subsec:homeplug}

Im Gegensatz zu Automotive Ethernet existiert zwischen Fahrzeug und Ladestation mit dem Control Pilot Pin eines IEC 62196-2 Ladesteckers nur eine Datenleitung für Steuersignale. Auf diesem Pin liegt ein Signal mit einer Frequenz von 1 kHz und einer Spannung von +/-12 V an, die je nach Akkuzustand über einen größeren oder kleineren Widerstand auf der Fahrzeugseite auf den Schutzleiter abgeleitet wird \cite[]{ladestecker}.

Weiterer Datenverkehr wird durch ein Nutzsignal mittels orthogonalem Frequenzmultiplexverfahren auf die Frequenz des Control Pilot Pins moduliert. Dieses Verfahren ist vielen durch die Verwendung in Powerline Netzwerkadaptern, häufig nur als DLAN-Steckern bezeichnet, bekannt, welche häufig auf den Standard HomePlug AV setzen. Für die Verwendung zwischen einem Fahrzeug und einer Ladesäule wird laut ISO 15118-3 „Physical and data link layer requirements“ der Standard HomePlug GreenPhy verwendet, welcher zu HomePlug AV kompatibel ist, allerdings einen geringeren Funktionsumfang bietet \cite[S. 302]{homepluggreenphy}.


Aufgrund von Übersprecheffekten anderer Ladesäulen über das gemeinsame Stromnetz, kann es vorkommen, dass dem Elektrofahrzeug mehrere Ladestationen auf eine Kommunikationsaufforderung antworten. Das SLAC-Protokol hat die Aufgabe, die Ladestation zu finden, mit der das Fahrzeug physikalisch verbunden ist. Dafür sendet das Fahrzeug zunächst eine \path{CM_SLAC_PARM.REQ} mit entsprechenden Parametern. Die Ladestationen antworten darauf mit einer \path{CM_SLAC_PARM.CNF} Nachricht, die Parameter wie die Anzahl der M-Sounds angibt. Daraufhin signalisiert das Fahrzeug mit \path{CM_START_ATTEN_CHAR.IND} Nachrichten, dass die Ladestationen mit der Messung der Signalabschwächung beginnen sollen. Die tatsächliche Messung der Signalabschwächung geschieht anhand der \path{CM_MNB_SOUND.IND} des Fahrzeugs, welches entsprechend M-Sounds oft versendet wird. Auf jede dieser Nachrichten antwortet die Ladestation mit einer \path{CM_ATTEN_PROFILE.IND} Nachricht, welche die gemessene Signalabschwächung enthält. 

Anhand der gemessenen Werte bestimmt dann das Fahrzeug die richtige Ladesäule, was es über eine \path{CM_SLAC_MATCH.REQ} Nachricht der Ladesäule signalisiert, welche darauf mit einer \path{CM_SLAC_MATCH.CNF} Nachricht antwortet. 

Die Antwort der Ladesäule enthält die Network-ID und den Network-Membership-Key, welchen das Fahrzeug nutzt, um dem logischen Netzwerk der Ladesäule beizutreten und die Kommunikation auf einer höheren Schicht fortzuführen \cite[S. 310]{homepluggreenphy}.

\begin{figure}
	\centering
	\includegraphics[width=\textwidth]{images/SLACC.png}
	\caption{Antwort der Ladesäule mit der Stärke des Signals auf eine Sound-Anfrage des Fahrzeugs}
	\label{fig:slacc}
\end{figure}

\subsubsection{Wi-Fi}\label{subsec:wifi}

Neben kabelgebundener Datenübertragung erlaubt ISO 15118-20 auch induktives Laden. Dabei findet die Datenübertragung anstelle von HomePlug über IEEE802.11n Wi-Fi 4 statt. \cite{ladepause} kritisiert an ISO 15118-20, dass zwar die Kommunikation mit TLS-Version 1.2 oder höher abzusichern ist, allerdings keine WPA2 Verschlüsselung gefordert wird. Ein alternativer Ansatz, der die Datenübertragung per IEEE802.11p vornimmt und die Datensicherheit mit ETSI TS 103 097 herstellt, wird zurzeit in Fachkreisen diskutiert.

\subsubsection{IPv6 Neighbor Discovery Protocol}\label{subsec:ipv6ndp}

Um die Ladesäule und das Fahrzeug in ihrem Link-Local IPv6 Netzwerk aufzulösen, verwenden sie das Neighbor Discovery Protocol. Dazu wird ein Neighbor Discovery Cache auf den Geräten angelegt, der IPv6 Adresse mit Ethernet Adressen (MAC Adressen) verknüpft (vergleichbar mit dem ARP-Cache bei IPv4). Dies geschieht über die folgenden ICMPv6 Nachrichten:

\begin{itemize}
\item \path{Router Solicitation}: Werden von Hosts verschickt, um einen Router aufzufordern, ein \path{Router Advertisement} zu senden
\item \path{Router Advertisement}: Wird regelmäßig vom Router oder auf Anfrage eines Hosts versendet und enthält verbindungsspezifische Informationen wie Routen
\item \path{Neighbor Solicitation}: Werden von Hosts verschickt, um die MAC-Adresse eines anderen Hosts anzufordern, aber auch für Funktionen wie die Erkennung doppelter Adressen (Siehe Kapitel \ref{subsec:slaac})
\item \path{Neighbor Advertisement}: Werden als Antwort auf \path{Neighbor Solicitation}-Nachrichten gesendet. Wenn ein Host seine MAC-Adresse ändert, kann er unaufgefordert diese Nachricht senden, um die neue Adresse bekannt zu geben.
\end{itemize}

\cite[S. 499]{ipnetze}

\subsubsection{IPv6 Stateless Address Autoconfiguration}\label{subsec:slaac}
Anders als in IPv4 benötigt IPv6 nicht zwingend einen DHCP Server, um eine IP-Adresse zu erhalten. 

Dazu verwendet die Ladestation oder das Fahrzeug das Link-Local Präfix fe80:: für die ersten 64 Bits der Adresse, invertiert die ersten 24 Bits der MAC-Adresse, fügt die 16 Bits \path{0xfffe} hinzu und vervollständigt die Adresse mit den letzten 24 Bits der MAC-Adresse. Daraufhin überprüft das Gerät mit einer \path{Neighbor Solicitation} Nachricht, ob ein anderes Gerät bereits diese Adresse verwendet.
Im eigentlich ausgeschlossenen Fall, dass ein Gerät mit einem \path{Neighbor Advertisement} antwortet (MAC-Adressen dürfen nicht doppelt vergeben werden), kann der Netzwerkteilnehmer versuchen, erneut eine Adresse zu generieren und prüft diese erneut auf eine Kollision mit einem \path{Neighbor Advertisement} \cite[S. 505]{ipnetze}.


\subsubsection{SECC Discovery Protocol}\label{subsec:secc}

Nachdem der DataLink etabliert wurde, sendet das Fahrzeug einen \path{SECCDiscoveryReq} als IPv6 Multicast an alle Netzknoten. In dieser Nachricht fragt das Fahrzeug nach der IP-Adresse und dem Port der Ladesäule, mit der es sich über TCP verbinden soll. Darin legt es fest, ob die Verbindung über TLS verschlüsselt werden soll oder nicht. Zusätzlich enthält es noch ein Byte, mit dem das Transportprotokoll festgelegt wird. Dieses ist jedoch laut \cite[]{masterIso15118} aktuell unbenutzt, da die Kommunikation im Standard nur über TCP stattfindet. 

Die Ladesäule antwortet darauf mit einer \path{SECCDiscoveryRes} Nachricht, welche die IP der Ladesäule, den Port für die TCP Verbindung und ob die Ladesäule TLS unterstützt enthält.

Bestimmte Funktionen wie Plug-and-Charge können nur verwendet werden, wenn die Verbindung TLS verschlüsselt ist. In der Version zwei des Standards ist TLS für die Kommunikation verpflichtend. 

\cite[S. 123]{masterIso15118}.

\subsubsection{Transport Layer Security}\label{subsec:tls}

Bei TLS handelt es sich um ein Ende-zu-Ende-Verschlüsselungsprotokol, das besonders empfindliche Daten wie die Zahlungsinformationen bei Plug-and-Charge einer Ladesäule absichern soll. Der Standard ist vorwiegend durch seine Verwendung im Internet bekannt, wird aber in der AUTOSAR Classic Plattform seit Release 4.4.0. unterstützt \cite[Kapitel 5.2]{automotiveprotocolssecurity}.

Das TLS-Protokoll war ursprünglich als SSL (Secure Socket Layer) bekannt und wurde von Netscape entwickelt. Die dritte Version von SSL ist heute als TLS 1.0 bekannt und wird in RFC 2246 beschrieben. Die aktuellste Version des Standards ist TLS 1.3, das im August 2018 als RFC 8446 veröffentlicht wurde \cite[S. 376]{ipnetze}.

Um TLS gemäß ISO 15118 Version 1 zu verwenden, müssen Ladesäule und Fahrzeug TLS 1.2 sowie mindestens die folgenden zwei TLS Chiffren unterstützen:
\begin{itemize}
\item \path{TLS_ECDH_ECDSA_WITH_AES_128_CBC_SHA256}\footnote{\url{https://ciphersuite.info/cs/TLS_ECDH_ECDSA_WITH_AES_128_CBC_SHA256/}}
\item \path{TLS_ECDHE_ECDSA_WITH_AES_128_CBC_SHA256}\footnote{\url{https://ciphersuite.info/cs/TLS_ECDHE_ECDSA_WITH_AES_128_CBC_SHA256/}}
\end{itemize}

Laut der zweiten Version des ISO 15118 Standards, sollen TLS 1.3 und die folgenden zwei TLS Chiffren verwendet werden:
\begin{itemize}
\item \path{TLS_ECDH_ECDSA_WITH_AES_128_GCM_SHA256}\footnote{\url{https://ciphersuite.info/cs/TLS_ECDH_ECDSA_WITH_AES_128_GCM_SHA256/}}
\item \path{TLS_ECDHE_ECDSA_WITH_AES_128_GCM_SHA256}\footnote{\url{https://ciphersuite.info/cs/TLS_ECDHE_ECDSA_WITH_AES_128_GCM_SHA256/}}
\end{itemize}

\cite[]{frauenhoferIsoTLS}

\begin{figure}[H]
	\centering
	\includesvg[width=\textwidth,inkscapelatex=false]{images/TLS_Ladesäule.svg}
	\caption{Sequenz der TLS Kommunikation \\ Quelle: Eigene Darstellung nach \cite{masterIso15118}}
	\label{fig:context}
\end{figure}

Die Plug-and-Charge-Erweiterung von ISO 15118 ermöglicht es Fahrzeugen, sich an einer Ladestation ohne weitere Authentifizierungsverfahren wie eine App oder RFID-Karte zu authentifizieren. Dazu benötigen die Ladesäule und das Fahrzeug ein digitales X.509v3-Zertifikat. Dieses sogenannte Leaf-Zertifikat wird von einem OEM mit einem Sub-CA Zertifikat bei der Herstellung des Fahrzeugs und der Ladesäule ausgestellt. Die Sub-CA Zertifikate der OEMs werden wiederum von einem Root Zertifikat, dem sogenannten V2G Root, ausgestellt. Wer welche Zertifikate ausstellen darf, ist im Handbuch „Zertifikats-Handhabung für Elektrofahrzeuge, Ladeinfrastruktur und Backend-Systeme im Rahmen der Nutzung von ISO 15118“ des Deutschen Verbands der Elektrotechnik, Elektronik und Informationstechnik geregelt \cite[S. 10]{pkiInfo}.

Das Fahrzeug kann mit dem Online Certificate Status Protocol überprüfen, ob ein Zertifikat einer Ladesäule widerrufen wurde. Eine Prüfung der Zertifikate des Fahrzeugs findet in diesem Schritt nicht statt \cite[S. 10]{pkiInfo}. 

Mit der OCSP Prüfung der Ladesäulen Zertifikaten läuft der TLS Verbindungsaufbau folgendermaßen ab:

\begin{enumerate}
\item Das Fahrzeug sendet im \path{ClientHello} Handshake seine Verbindungsparameter wie die unterstützen Chiffren an die Ladesäule. Durch die \path{status_request_v2} Extension fordert das Fahrzeug die Ladesäule auf, ihre Zertifikate durch einen OCSP Request zu authentifizieren. Zudem sendet das Fahrzeug mit der \path{trusted_ca_keys} Extension eine Lister aller V2G Root Zertifikaten, denen es vertraut.
\item Die Ladesäule verarbeitet das \path{ClientHello} 
\item Die Ladesäule sendet eine OCSP Anfrage mit seinen Zertifikaten an den OCSP Responder. 
\item In der OCSP Antwort wird mit einem Zeitstempel angegeben, ob die überprüften Zertifikate widerrufen wurden. Die Antwort wird zusätzlich digital signiert, um die Authentizität der Antwort zu beweisen.
\item Die Ladesäule antwortet mit dem \path{ServerHello} seine Verbindungsparameter wie die höchstmögliche unterstützte Chiffre. Mit der \path{status_request_v2} Extension signalisiert die Ladesäule OCSP Validierungen zu unterstützen.
\item Die Ladesäule sendet seine Zertifikatskette, bestehend aus Sub-CA und Leaf. Diese müssen von einem aus Schritt 1 unterstützten V2G Root Zertifikat abgeleitet sein.
\item Die OCSP Antwort aus Schritt 4 wird an das Fahrzeug übertragen
\item Entsprechenden der gewählten Chiffre werden Informationen für die Bildung eines gemeinsamen Schlüssels versandt
\item Die Ladesäule signalisiert, dass die „Hello“-Phase abgeschlossen ist
\item Das Fahrzeug validiert das Zertifikat der Ladesäule und die OCSP Antwort
\item Das Fahrzeug sendet nun wie die Ladesäule in Schritt 8 Informationen, um einen gemeinsamen Schlüssel zu bilden
\item Das Fahrzeug teilt mit, dass ab nun die Kommunikation verschlüsselt stattfindet
\item Das Fahrzeug teilt mit, das der Schlüsselaustausch und die Authentifizierung erfolgreich waren
\item Die Ladesäule teilt mit, dass ab nun die Kommunikation verschlüsselt stattfindet
\item Die Ladesäule teilt mit, das der Schlüsselaustausch und die Authentifizierung erfolgreich waren
\end{enumerate}

\subsubsection{Vehicle to Grid Transfer Protocol}\label{subsec:v2g}

Bei dem V2GTP Protokoll handelt es sich um einen Session Wrapper, der die Daten mit Headern versieht. Diese Header enthalten Informationen über die Version, die Länge und der Kodierungsmethode der Payload \cite[S. 27]{masterIso15118}.

\subsubsection{Vehicle to Grid Protocol}\label{subsec:v2g}

Die eigentlichen Daten werden im Efficient XML Interchange Format nach den in ISO 15118-2 definierten Schemata formatiert.
Bei dem Efficient XML Interchange Format handelt es sich um XML, das in Binär anstelle von Klartext gespeichert wird. Dies reduziert zum einen die Dateigröße, beschleunigt aber auch das Parsen \cite[]{exi}.

\section{Vector Informatik GmbH}\label{subsec:vectorunternehmen}

Die Vector Informatik GmbH ist einer der führenden Hersteller von Softwarewerkzeugen, Hardware und Softwarekomponenten für die Entwicklung elektronischer Systeme und deren Vernetzung über die in Kapitel \ref{subsec:kommunikationsstacks} genannten Busse. Dabei ist das Unternehmen insbesondere in der Automobilbranche präsent. Am 1. April 1988 gründeten Martin Litschel, Dr. Helmut Schelling und Eberhard Hinderer die „Vector Software GmbH“ in Ditzingen. Das erste Projekt des Unternehmens war damals eine Softwarelösung für NC-gesteuerte Bearbeitungszentren, die aufgrund komplexer Vektorberechnungen auch als Inspiration für den Firmennamen diente. Im Jahr 1995 zog das Unternehmen von Ditzingen nach Stuttgart-Weilimdorf. 1997 erfolgte die Gründung der ersten Auslandstochter, der Vector CANtech in Novi (USA). Heute ist das Unternehmen an 26 Standorten in 13 Ländern mit über 3.000 Mitarbeitern weltweit tätig und erzielte 2019 einen Umsatz von 770 Millionen Euro \cite[]{firmengeschichte}.

\section{CANoe}\label{subsec:canoe}

CANoe\footnote{\url{https://www.vector.com/canoe/}} ist ein Software-Entwicklungs- und Testwerkzeug der Vector Informatik GmbH, das vorwiegend von Automobilherstellern und -zulieferern für die Entwicklung, Simulation, dem Testen und der Inbetriebnahme von Kommunikationsnetzwerken und Steuergeräten eingesetzt wird. Steuergerätesoftware kann zu diesem Zweck entweder virtualisiert oder über entsprechende Schnittstellen auf reale Hardware in einer Testumgebung ausgeführt werden. In dieser Testumgebung kann die Testfunktionalität mit der ereignisorientierten Programmiersprache CAPL oder CSharp implementiert werden. Die Kommunikation in der Testumgebung kann mit umfangreichen Traces nachverfolgt und somit debuggt werden. CANoe unterstützt alle in Kapitel \ref{subsec:kommunikationsstacks} genannten Kommunikationsstacks. 

Für Simulationen unterscheidet CANoe zwischen den folgenden zwei Arbeitsmodi:
\begin{itemize}
\item \textbf{Simulierter Bus} Die Simulation findet in einem komplett simulierten Netzwerk statt. Es ist möglich, die Geschwindigkeit einer Simulation zu verändern.
\item \textbf{Realer Bus} Die Simulation findet auf einem verbundenem Netzwerkinterface statt. Dies ermöglicht das Hinzufügen realer Hardware.
\end{itemize}

\cite[]{simulatingcanoe}

\section{vTESTstudio}\label{subsec:vteststudio}

vTESTstudio\footnote{\url{https://www.vector.com/vteststudio/}} ist eine Testumgebung der Vector Informatik GmbH für eingebettete Systeme, die das Design, die Erstellung und die Ausführung von Tests erleichtert. Dazu können in der Entwurfs- und Implementierungsphase abstrakte Testfälle in CAPL, CSharp oder Python implementiert werden, die in Menüs für Testfälle entsprechend zusammengestellt und parametrisiert werden können. Für die Entwicklung von Elektroladesäulen bietet Vector mit dem CANoe Test Package EV\footnote{\url{https://www.vector.com/de/de/produkte/produkte-a-z/software/canoe-test-package-ev/}} vorgefertigte Testfälle an, die vom Kunden in vTESTstudio konkret parametrisiert werden können. 

Die Tests werden in der Testphase in für CANoe verständlichen Code kompiliert und in der Simulation ausgeführt. Aus den ausgeführten Tests können Reports für die Testdokumentation generieren. 

\section{Metasploit Framework}\label{subsec:metasploit}

Im Jahr 2003 begann der amerikanische Sicherheitsexperte H. D. Moore mit der Entwicklung des Metasploit-Projekts\footnote{\url{https://www.metasploit.com/}}. Die Grundidee hinter der Anwendung war damals, wie heute, der Öffentlichkeit Zugang zu detaillierten Informationen über Sicherheitslücken zu verschaffen, was nach wie vor die Frage nach dem richtigen Vorgehen bei der Veröffentlichung von Sicherheitslücken aufwirft \cite[S.45]{Messner2017}.

Seit der Version 3 des Projekts ist die Anwendung in Ruby verfasst und vereint als Framework eine Vielzahl an Tools. Das reine Open Source Framework kann über die Kommandozeilenanwendung \texttt{msfconsole} oder über die von \texttt{msfrpc} bereitgestellte API bedient werden. Die Pro-Version der Anwendung ermöglicht es zusätzlich, die Anwendung über ein Webinterface zu bedienen.

\subsection{Module}\label{subsec:metasploitmodule}
Unter Modulen versteht man im Kontext von Metasploit ein Stück Software, das parametrisiert werden kann und bei der Ausführung eine bestimmte Aufgabe ausführt. Eingeteilt werden die Module in die folgenden Kategorien:

\begin{itemize}
  \item \textbf{Auxiliary} Führt beliebige Aktionen durch, die nicht mit der Ausnutzung einer Schwachstelle in Verbindung stehen
  \item \textbf{Exploits} Nutzt eine Sicherheitslücke in einem System oder einer Anwendung aus
  \item \textbf{Payloads} Code, der ausgeführt wird, nachdem ein Exploit ein System erfolgreich kompromittiert hat
   \item \textbf{NOPs} Code, der nichts tut, um den Speicher für Exploits entsprechend vorzubereiten
  \item \textbf{Encoders} Kodieren eine Payload oder Nop so, dass sie von Schutzmaßnahmen nicht erkannt werden
\end{itemize}
\cite[]{metasploitglosar}

\subsection{Jobs}\label{subsubsec:auxiliarys}

Aufgaben, wie das Ausführen eines Moduls, können in Metasploit mithilfe eines Jobs durchgeführt werden. Dafür wird vor dem Start des Jobs dieser entsprechend konfiguriert, sodass er z. B. Konsolenausgaben in einer Datei ablegt. Jeder Job erhält beim Erstellen eine eindeutige ID, über die er gestartet oder gestoppt werden kann. 

\section{Scapy}\label{subsec:scapy}
Bei Scapy handelt es sich um eine Python Bibliothek, welche eine Vielzahl an Funktionalitäten für die Manipulation von Netzwerkpaketen liefert. Dafür wird in Python definiert, wie ein Paket aufgebaut ist, was das Parsen und das Bearbeiten vereinfacht. Zusätzlich werden beim Bearbeiten von Paketen Abhängigkeiten wie z. B. Checksummen selbstständig neu berechnet \cite{scappy}.

In den Arbeiten von \cite{scapyAutomotive} wird Scapy um Paketdefinitionen\footnote{\url{https://scapy.readthedocs.io/en/latest/layers/automotive.html}} bekannter Automobilprotokolle ergänzt, was eine tiefe Untersuchung dieser erlaubt.
\chapter{Anforderungen}\label{sec:anforderungen}
Mit der immer weiter anhaltenden Digitalisierung von Fahrzeugen steigt die Gefahr von Cyberangriffen auf Fahrzeuge. Aufgrund dieser Entwicklung fordern die Regulierungsbehörden zunehmend Maßnahmen für die Zulassung von Fahrzeugen, um den Schutz vor Angriffen dieser Art zu gewährleisten. Ein Pentest kann dabei eine Maßnahme sein, um den Anforderungen gerecht zu werden. 

\section{Pentesting Prozess}\label{sec:generalpentesting}

Laut dem \cite{pentest} kann ein Pentest dazu dienen, Schwachstellen in einem IT-System zu identifizieren oder dazu genutzt werden, die IT-Sicherheit eines Systems durch einen externen Dritten zu bestätigen. Es ist jedoch zu beachten, dass trotz eines Pentests noch Schwachstellen in einem System vorhanden sein können, die beispielsweise zum Zeitpunkt des Pentests noch nicht bekannt waren. 

Für ein strukturiertes Vorgehen bei einem Pentest schlägt das BSI die folgenden fünf Phasen als Vorgehensweise vor:

\subsection{Vorbereitung}\label{subsec:pentestvor}
Vor dem eigentlichen Pentest muss mit dem Auftraggeber geklärt werden, welche Ziele damit erfüllt werden sollen. Je nach Zielsetzung hat dies einen starken Einfluss auf das Vorgehen bei einem Pentest. Der Auftraggeber kann zum Beispiel verlangen, dass beim Testen eines Produktivsystems die Funktionalität nicht beeinträchtigt wird, was besondere Sorgfalt bei der Auswahl der Exploits erfordert. 
Zudem muss von jeder Partei eines Systems die Zustimmung zum Pentest eingeholt werden, da ein unbefugter Angriff eine Straftat darstellen kann.

\subsection{Reconnaissance}\label{subsec:informationsbeschafung}

Ziel dieser Phase ist es, einen möglichst umfassenden Überblick über ein System und mögliche Angriffspunkte zu erhalten. In Metasploit sind Module dieser Phase als \path{auxiliaries} bekannt (siehe Kapitel \ref{subsec:metasploitmodule}).

\subsection{Assessment}\label{subsec:risikoanalyse}

Bevor ein Eindringversuch gestartet wird, werden die vereinbarten Ziele des Penetrationstests, die potenzielle Gefährdung des Systems und der geschätzte Aufwand analysiert. Auf der Grundlage dieser Analyse werden die Ziele für die folgende Phase ausgewählt. 

\subsection{Exploitation}\label{subsec:eindringversuch}

In dieser Phase wird das System aktiv angegriffen. So kann geprüft werden, inwieweit die vermeintlichen Schwachstellen ein reales Risiko darstellen. Metasploit nennt Module dieser Phase \path{exploits} (siehe Kapitel \ref{subsec:metasploitmodule}).

\subsection{Abschlussanalyse}\label{subsec:eindringversuch}

Zuletzt wird die Dokumentation für alle vorangegangenen Phasen erstellt. Es sollte dabei darauf geachtet werden, dass die Schwachstellen für den Auftraggeber nachvollziehbar sind. Ferner sollten konkrete Empfehlungen für die Behebung der gefundenen Schwachstellen gegeben werden.

\section{Rechtlicher Rahmen}\label{sec:automotivepentesting}

Im Jahr 2018 begann eine Arbeitsgruppe der Wirtschaftskommission der Vereinten Nationen für Europa (UNECE) und der Internationalen Organisation für Normung mit der Untersuchung einheitlicher Regulierungssysteme für die Cybersicherheit in der Fahrzeugherstellung. Die Ergebnisse beider Organisationen wurden ca. 2021 veröffentlicht und als Grundlage für Verordnungen der Europäischen Kommission verwendet. Ab Juli 2022 müssen alle neuen Fahrzeugtypen die Anforderungen der UNECE-Regelung Nr. 155 erfüllen und ab Juli 2024 alle neu produzierten Fahrzeuge (siehe Abbildung \ref{fig:regulierungen}).

\begin{figure}[H]
	\centering
	\includesvg[width=\textwidth,inkscapelatex=false]{images/Regulierungen.svg}
	\caption{Standards zur Cybersecurity im Automobilsektor \\ Quelle: Eigene Darstellung nach \cite[]{Ebert2021}}
	\label{fig:regulierungen}
\end{figure}

\subsection{UNECE Regulation No. 155}\label{subsec:unce}
Bei der UNECE Regulation No. 155 „Proposal for a new UN Regulation on uniform provisions concerning the approval of vehicles with regards to cyber security and cyber security management system“ handelt es sich um eine Verordnung der „United Nations Economic Commission for Europe“, welche von Automobilherstellern ein Cybersecurity-Management-System (CSMS) verlangt. Ab Juli 2024 ist sie für sämtlichen Neuwagen in der EU verpflichtend.  

Das CSMS soll über die in § 7.2.2.2. genannten Verfahren sicherstellen, dass die Hersteller über entsprechende Sicherheitsmaßnahmen in der Entwicklungs-, Produktions- sowie Postproduktionsphase verfügen, um die folgenden Anforderungen zu erfüllen:
 \begin{quote}
\begin{enumerate}[label=\alph*.]
  \item Erfassung und Überprüfung der gesamten Lieferkette hinweg, um nachzuweisen, dass lieferantenbezogene Risiken ermittelt und bewältigt werden
  \item Dokumentation der Risikobewertung (während der Entwicklungsphase oder nachträglich), der Testergebnisse und der Minderungsmaßnahmen bezogen auf den Fahrzeugtyp, einschließlich konstruktionsbezogener Informationen zur Untermauerung der Risikobewertung
  \item Implementierung geeigneter Cybersicherheitsmaßnahmen bei der Konzeption des Fahrzeugtyps
  \item Erkennung von und Reaktion auf mögliche Cyberangriffe
  \item Protokollierung von Daten zur Unterstützung der Erkennung von Cyberangriffen und Bereitstellung von Datenforensik, um eine Analyse versuchter oder erfolgreicher Cyberangriffe zu ermöglichen
\end{enumerate}
\cite[§ 7.2.2.2.]{un115}
\end{quote}

Eine Institution wie die GTÜ ATEEL AG\footnote{\url{https://www.gtue.de/de/die-gtu/unsere-tochterunternehmen/ateel}} überprüft alle drei Jahre, ob das vom Hersteller vorgeschlagene CSMS die genannten Punkte erfüllt und stellt ihm in diesem Fall eine Konformitätsbescheinigung für sein CSMS aus.

Konkrete Empfehlungen für Maßnahmen werden im „Proposal for the Interpretation Document for UN Regulation No. 155 on uniform provisions concerning the  approval of vehicles with regards to cyber security and cyber security management system“ genannt.

Um Punkt E „Verfahren zum Testen der Cybersicherheit eines Fahrzeugtyps“ in der Entwicklungsphase zu erfüllen, schlägt das Proposal unter anderem die Norm ISO/SAE 21434 als Leitlinie vor.
 
\subsection{ISO/SAE 21434}\label{subsec:iso21434}

In ISO/SAE 21434 werden für jede Phase des Produktzyklus organisatorische Maßnahmen vorgegeben, die für die Zertifizierung durchgeführt werden müssen. Dafür wird hauptsächlich die Aufgaben jeder Maßnahme beschrieben, was es erlaubt, die Umsetzung der Maßnahmen weitestgehend selbst zu bestimmen. 

Penetration Tests sind laut RC-10-12 des Standards in der Entwicklungsphase als Methode empfohlen, um zu bestätigen, dass nicht identifizierte Schwachstellen auf ein Minimum reduziert wurden.

In der Validierungsphase sind sie laut RQ-11-01 eine geeignete Methode, um zu zeigen, dass ein adäquates Sicherheitsniveau erreicht wurde.
 
\section{Integration in den Entwicklungsprozess}\label{fig:vmodell}

\begin{figure}[H]
	\centering
	\includegraphics[width=\textwidth]{images/V-Modell.pdf}
	\caption{Grey Box Pentesting in V-Modell Automotive-Projekten \\ Quelle: Eigene Darstellung nach \cite[]{Ebert2021}}
	\label{fig:vmodel}
\end{figure}

\cite{Ebert2021} beschreiben in „Penetration Testing for Automotive Cybersecurity“ anhand des V-Modells, wie ein Pentest während der Definitions- und Entwicklungsphase (linke Seite des V-Modells) und der Test- und Integrationsphase (rechte Seite des V-Modells) durchgeführt werden kann. Dafür werden die folgenden 10 Schritte während des in Abbildung \ref{fig:vmodel} dargestellten Entwicklungsprozesses durchgeführt:

\begin{enumerate}
    \item \textbf{Reconnaissance:} Überprüfen von Featuren der Anwendung, ob diese missbraucht werden können
    \item \textbf{Structural analysis:} Bestimmen der Komponenten der Software, die im Pentest überprüft werden sollen
    \item \textbf{Asset elicitation:} Erfassen von Teilen der Komponenten, die überhaupt schützenswert sind
    \item \textbf{Specifications:} Recherche und Scannen nach bekannte Schwachstellen in der Hard- und Softwarespezifikation
    \item \textbf{TARA, security goals:} Durchführen einer Bedrohungsanalyse und Risikobewertung (TARA)
    \item \textbf{Minimum Viable Pentest Cases:} Erstellen einer Testumgebung, in der ein Pentest ausgeführt werden kann
    \item \textbf{Penetration Testing:} Durchführen von Pentests in der Testumgebung und Dokumentation der Ergebnisse
    \item \textbf{Key performance indicators:} Prüfen der Effektivität des Pentests anhand von KPI Metriken wie der Threat coverage, security test efficiency, vulnerability detection effectiveness
    \item \textbf{Functional security requirements:} Aus den gewonnenen Erkenntnissen des Pentests erneut Komponenten identifizieren, die überprüft werden sollten
    \item \textbf{Regression testing:} Bei Codeänderungen muss sichergestellt werden, dass nicht neue Schwachstellen eingeführt wurden
\end{enumerate}

\subsection{Tooling Support für Durchführung eines Pentests}\label{subsec:fallbeispiel}

Um den Mitarbeitenden bei der Entwicklung Penetrationstests zugänglich zu machen, benötigen sie entsprechende Softwaretools, die dies ermöglichen. 

Ideal für die Dokumentation der Bedrohungsanalyse und Risikobewertung ist die Anwendung COMPASS 2\footnote{\url{https://consulting.vector.com/de/de/solutions/compass-und-preevision/compass/}} von Vector, mit der sich systematisch Berichte erstellen lassen.

Als Prüfsystem eignet sich die Anwendung CANoe, mit der ein komplettes Steuergerät in einer simulierten Umgebung getestet werden kann. 

Für das eigentliche Pentesten eignet sich das etablierte Metasploit-Framework, das mit einer Vielzahl von Modulen, die über das Metasploit CLI parametrisiert werden können, schnell zu Ergebnissen führt.

Um die durchgeführten Pentests zu dokumentieren und für Regressionstests einfach wiederholen zu können, kann vTESTStudio eingesetzt werden. Darin können die ausgeführten Schritte des Pentests in einem Integrationstest automatisiert werden und die Ergebnisse der Metasploit-Module entsprechend ausgewertet werden.

Die grundlegenden Werkzeuge sind bereits vorhanden, allerdings fehlt eine Integration dieser untereinander. So existiert in Metasploit kein Mechanismus, um auf den Datenverkehr eines CANoe Netzwerks zuzugreifen. Des Weiteren ist es für den Endanwender sehr umständlich, während des Pentests gleichzeitig mit CANoe und Metasploit arbeiten zu müssen. Es wäre deshalb sinnvoll, die Bedienung Metasploits weitestgehend in CANoe zu ermöglichen. 

\section{Architektur der Anwendung}\label{subsec:context}

\subsection{Kontext der Anwendung}\label{subsec:kontext}
Die zu entwickelnde Anwendung wird im Rahmen eines Pentests verwendet, der Schritt 7 der im Abbildung \ref{fig:vmodell} vorgeschlagenen Prozesses ausführt. Als Grundlage dieses Pentests dient ein CANoe Projekt, welches während der Implementierung als Testumgebung gedient hat. Bei einem Gespräch mit dem Auftraggeber der Anwendung wurden die folgenden zwei Zielgruppen identifiziert:

\begin{itemize}
\item Security-ExpertInnen sollten in der Lage sein, eine Analyse mit etablierten Pentest-Tools und Frameworks durchzuführen, diese zu dokumentieren und in einem Modul für das erneute Ausführen aufbereiten
\item Automobil-EntwicklerInnen sollen aus einem breiten Katalog fertiger Pentest Module auswählen können und damit eigenständig eine Komponente überprüfen
\end{itemize}

Dabei sollen beide Gruppen in ihrer Domäne bleiben können, sprich die Security-ExpertInnen in ihrer Linux- und die Automobil-EntwicklerInnen in ihrer gewohnten CANoe-Umgebung. Die Verbindung zwischen diesen beiden Welten sollen die aus Metasploit bekannten Module darstellen. Die Dokumentation dieser soll es der CANoe EntwicklerIn eigenständig erlauben, passende Exploits zu finden und auszuprobieren. Allerdings benötigt die Automobil-EntwicklerInnen immer noch genug Fachkenntnisse, um zu verstehen, was in dem Modul vor sich geht, um das Modul entsprechend parametrisieren zu können. In diesem Punkt unterscheidet sich die Anwendung von Vulnerability Scanning Tools, die weitestgehend ohne manuelles Zutun nach Schwachstellen suchen können. 

\subsection{Container der Anwendung}\label{subsec:container}
Der Tester sollte nur mit dem CANoe System und den darin enthaltenen Containern interagieren müssen, um ein Pentestmodul auszuführen. Damit der PentestRunner mit dem gegebenen CANoe Projekt arbeiten kann, muss ihm eine entsprechende MAC- und IP-Adresse zugewiesen werden. Die GUI des PentestStudios wird über ein Panel in das CANoe Projekt hinzugefügt. In zukünftigen Versionen sollte dieses ähnlich wie der Plattform Manager Teil von CANoe werden, um es aus einem Menü einfach zu starten. 

In PentestStudio sollen Tester in der Lage sein, vorgefertigte Pentestmodule, z. B. aus ExploitDB oder speziell von einem Sicherheitsexperten vorbereitet, zu untersuchen, zu parametrisieren und auszuführen. 

Zusätzlich sollen sie aus diesen Modulen entsprechende Testmethoden generieren können, welche einfach in vTESTstudio Tests integriert werden können. Diese Tests können dann nach dem Beheben eines Befunds als Regressionstests dienen, um sicherzustellen, dass derselbe Fehler nicht erneut gemacht wird. 

Mit Metasploit sollte ein Sicherheitsexperte in der Lage sein, mit einem in seiner Expertise etablierten Tools zu arbeiten. Dies soll die größtmögliche Wiederverwendbarkeit bestehender Lösungen gewährleisten und die schnellstmögliche Reaktion auf das Bekanntwerden neuer Exploits ermöglichen. Aus der CANoe Umgebung können Modulen als Parameter entsprechende Systemvariablen übergeben werden. 

\begin{figure}
	\centering
	\includesvg[width=\textwidth,inkscapelatex=false]{images/2_containers.svg}
	\caption{Container der Anwendung}
	\label{fig:context}
\end{figure}

\chapter{Implementierung}\label{sec:implementierung}

% Kritisch die Entscheidungsfindung hinerfragen und begründen warum etwas so und so umgesetzt wurde


\section{Zugriff auf das CANoe Ethernet Netzwerk}\label{sec:canoetcpip}
Blub
% Entscheidungen in der Implementierung begründen, warum wurde etwas so umgesetzt wie es umgesetzt wurde. 
% Benchmarking (ist es optimal)
\chapter{Evaluation}\label{sec:evaluation}
\section{Blub}\label{sec:angriffsziele}
\chapter{Fazit und Ausblick}\label{sec:zusammenfassung}

\section{Überblick}\label{sec:ueberblickfazit}

In diesem Projekt wurde eine Strategie zur Anbindung eines Exploit-Frameworks erforscht. Dazu wurde ein Prototyp einer Anbindung implementiert und anhand eines Pentests der ISO 15118 Ladekommunikation kontinuierlich evaluiert und erweitert. 

Durch die Zweckentfremdung der tapserver-Funktionalität von CANoe konnte eine zuverlässige Verbindung zwischen Linux-Anwendungen und CANoe-Netzwerken hergestellt werden. Docker erwies sich als die Umgebung der Wahl für die Ausführung der Linux-Anwendungen, vordergründig wegen seines hohen Automatisierungsgrades. Dank dieses hohen Grades konnte die Anbindung von Metasploit so implementiert werden, als wäre es nur eine weitere Komponente von CANoe. Durch die Verwendung von Docker-Netzwerktreibern konnte eine komplexe Netzwerkkonfiguration geplant und konfiguriert werden. Metasploit wurde um Funktionalitäten wie den Export der Ergebnisse einer Modulausführung in eine Redis-Datenbank erweitert, wodurch die Auswertung der ausgeführten Module ermöglicht wurde. Die Benutzeroberfläche von PentestStudio ermöglicht eine umfassende Fernsteuerung von Metasploit, ohne die Oberfläche von CANoe jemals verlassen zu müssen. Da das Ausführen von Modulen in PentestStudio komplett geloggt wird, können umfangreiche Reports über durchgeführte Untersuchungen erstellt werden, die als Grundlage weiterer Anwendungen verwendet werden können. Der Code-Generator von PentestStudio ermöglicht eine schnelle Integration von Modulen in die vTESTstudio-Testumgebung, was die Dokumentation von Schwachstellen und die Erstellung von Regressionstests erheblich erleichtert. 

Die in Metasploit enthaltenen Module sind für Automotive Pentesting kaum verwendbar und eignen sich hauptsächlich für bekannte Schwachstellen von Serveranwendungen, die in Automobilnetzen für gewöhnlich nicht vorkommen. Allgemeine Module wie Port-Scanner erwiesen sich im Vergleich zu bekannten Open-Source-Tools stets als funktional unterlegen, weshalb Kommandozeilen Tools wie nmap oder thc-ipv6 als Metasploit Modul verpackt wurden. 

Die selbst entwickelten Metasploit-Module ermöglichen einfache Angriffe auf die Ladekommunikation des ISO 15118 Standards. Durch die konkrete Nutzung der prototypisch implementierten Anwendung für das Pentesting konnte während der Implementierung kontinuierlich Feedback gesammelt werden. 

\section{Erweiterungsmöglichkeiten}\label{sec:erweiterungsmoeglichkeiten}

Zum einen wäre es möglich, die Installation von Docker Desktop oder Podman in das Installationsprogramm von CANoe einzubinden, was dem Endanwender die zusätzliche Installation der Anwendung ersparen würde. Derzeit erfolgt die Einrichtung des Docker-Containers über eine Docker Compose-Datei. Dieser Vorgang könnte manuell in einem Dialog der Anwendung oder automatisiert aus den Einstellungen eines CANoe Projekts erfolgen.

Mit der Weiterentwicklung der Tapserver-Funktionalität könnte es in Zukunft möglich sein, CAN, LIN, etc. über die Anwendung mit dieser oder einer ähnlichen Funktion zu testen.

Die Zuweisung einer Payload für einen Metasploit-Exploit ist in der aktuellen Version der Anwendung umständlich, da zwei Module nacheinander geladen werden müssen. In Zukunft könnte die Konfiguration in der Benutzeroberfläche entsprechend visualisiert werden. Das Auffinden eines Metasploit-Moduls erfordert derzeit die Verwendung einer Webanwendung wie \url{www.exploit-db.com} und \url{www.rapid7.com/db/} oder die Erstellung einer Ordnerstruktur mit gespeicherten PentestStudio-Projekten. Das Archiv der installierten Exploits könnte in einer zukünftigen Version mit einem Exploit-Browser visualisiert werden. 

%Um häufig verwendete Exploit-Module schneller laden zu können, wäre unter dem Menüpunkt \path{File} ein Eintrag \path{Recent Modules} sehr hilfreich.

Das Tool verlässt sich auf Metasploit als zentrale Ausführungsumgebung. Um andere Module nutzen zu können, wurde Metasploit um Funktionalitäten wie den \path{CLI-Runner} erweitert. In einer zukünftigen Version könnte PentestStudio möglicherweise eine Abstraktionsschicht auf einem Linux-Rechner steuern, die wiederum Metasploit steuern könnte. Für Anwendungsfälle, die nicht auf Metasploit angewiesen sind, würde dies die Entwicklung von Tools vereinfachen, da es nicht notwendig wäre, das Schema eines Metasploit-Moduls zwanghaft zu befolgen.

PentestStudio ist derzeit nur ein Panel in einem CANoe Projekt. Es wäre sinnvoll, die Anwendung als Teil von CANoe zu integrieren. Der Report der ausgeführten Module ist derzeit nur ein JSON und sollte mehr wie die Reports von vTESTstudio aussehen. 

Die entwickelten Metasploit-Module wurden größtenteils für den spezifischen Testfall dieser Arbeit entwickelt und müssten für die weitere Verwendung generischer gestaltet werden.

\section{Fazit}\label{sec:fazit}

Im Kontext von Fahrzeugen werden unter „Safety“ Merkmale verstanden, die verhindern sollen, dass ein Fahrzeug Schäden an sich selbst, der Umwelt oder am Fahrer verursacht. Bekannte Merkmale dieser Kategorie sind unter anderem der Airbag, das ABS oder das ESP. Unter „Security“ werden dagegen Merkmale verstanden, die dem Schutz des Fahrzeugs dienen sollen. Diese beiden Merkmale können jedoch nicht unabhängig voneinander betrachtet werden, da eine mangelhafte „Security“ auch die „Safety“ eines Fahrzeugs beeinträchtigen kann. Wird ein sicherheitskritisches Steuergerät durch einen Angriff kompromittiert, kann es im schlimmsten Fall seine „Safety“ Aufgaben nicht mehr erfüllen. Aus diesem Grund sollten „Security“ Merkmale immer als indirekte „Safety“ Merkmale betrachtet werden. Mit dieser Arbeit wurde der Grundstein für den verstärkten Einsatz von Pentesting als Validierungsmethode der „Security“ von Fahrzeugsoftware gelegt, wodurch auch zukünftig das hohe „Safety“-Niveau, das von modernen Fahrzeugen erwartet wird, gewährleistet werden kann.
%\chapter{Gestaltungsrichtlinien}\label{sec:gestaltungsrichtlinien}

\section{Sprache\index{Sprache} und Textumfag}\label{subsec:sprache_textumfang}

Laut Prüfungsordnung dürfen Bachelor- und Masterarbeiten in deutscher oder englischer Sprache verfasst werden.
Weitere Sprachen sind nach Absprache mit dem Prüfungsausschuss möglich.
Der geeignete Umfang von wissenschaftlichen Arbeiten ist abhängig von Thema, Schreibstil des Verfassers und Anzahl der Verfasser.
Generell sollte man sich an folgenden Richtwerten orientieren.
Die Werte beziehen sich auf den reinen Text ohne Inhaltsverzeichnis\index{Inhaltsverzeichnis}, Literaturverzeichnis oder Anhänge.

\begin{itemize}
    \item{Seminararbeiten ca. 20-30 Seiten}
    \item{Projektdokumentationen ca. 20-40 Seiten}
    \item{Bachelorarbeiten ca. 30-50 Seiten}
    \item{Masterarbeiten ca. 60-80 Seiten}
\end{itemize}

\section{Inhaltliche Bestandteile}\label{subsec:inhaltliche_bestandteile}

Jede Arbeit besteht aus den folgenden Bestandteilen in der vorgegebenen Reihenfolge.
Dabei beginnt jeder Abschnitt auf einer neuen Seite. 

\begin{enumerate}
    \item{Titelseite}
    \item{Inhaltsverzeichnis}
    \item{Abbildungsverzeichnis (optional)}
    \item{Tabellenverzeichnis (optional)}
    \item{Zusammenfassung/Abstract (bei Seminararbeiten und Projektdokumentationen optional)}
    \item{Einleitung}
    \item{Hauptteil}
    \item{Schluss}
    \item{Literaturverzeichnis}
    \item{Erklärung zur Urheberschaft}
    \item{Anhang (optional)}
    \item{Index (optional)}
\end{enumerate}

\minisec{Titelseite}\index{Titelseite}

Die Titelseite enthält alle wichtigen Angaben zu Lehrveranstaltung (Semester, Universität, Lehrveranstaltung, Dozent, Modul), den Titel der Arbeit (im genauen Wortlaut der Themenstellung), Angaben zum Verfasser (Name, Anschrift, E-Mailadresse, Matrikelnummer, Semesterzahl, Datum der Abgabe).
Zusätzlich werden bei Bachelor- und Masterarbeit Erst-und Zweitgutachter angegeben.

\minisec{Inahltsverzeichnis}

Das Inhaltsverzeichnis enthält alle Gliederungspunkte auf allen Ebenen mit den entsprechenden Seitenzahlen.
Es sollen nicht mehr als drei bis maximal vier Gliederungsebenen (1.1.1.1) verwendet werden. Die Nummerierung erfolgt in Dezimalgliederung (1., 1.1 usw.).
Inhaltsverzeichnis und Plagiatserklärung\index{Plagiatserklärung} werden im Inhaltsverzeichnis aufgeführt, aber nicht nummeriert.

\minisec{Einleitung}\index{Einleitung}

Die Einleitung muss nicht zwingend kreativ oder originell sein, sondern ergibt sich aus der Arbeit.
Da sich der genaue Inhalt und die Ergebnisse während der Bearbeitung ändern können, muss die Einleitung eventuell später angepasst werden.
Inhaltlich grenzt die Einleitung das Thema genau ein mit Formulierungen wie „diese Arbeit beschäftigt sich mit XY“.
Danach sollte aufgezeigt werden, inwiefern die Problemstellung für die Medieninformatik relevant ist, sowie die einzelnen Ziele der Arbeit.
Abschließend wird der inhaltliche Aufbau der Arbeit erläutert.
Allgemeinplätze wie „immer mehr Menschen verwenden Computer“ sollten unbedingt vermieden werden.

\minisec{Hauptteil}\index{Hauptteil}

Der Hauptteil beginnt mit einem kurzen Kapitel zu den Zielen der Arbeit.
Bei sehr kurzen Arbeiten, wie einer Seminararbeit, ist dieser Punkt meist mit der Einleitung schon abgedeckt.
Darauf folgt der aktuelle Forschungsstand zum Thema.
Dabei werden unterschiedliche Ansätze vorgestellt und ihre Vor- und Nachteile aufgezeigt.
Je nach Art des Themas fallen die weiteren inhaltlichen Bestandteile unterschiedlich aus.
Im Anhang\index{Anhang} A befinden sich die inhaltlichen Bausteine für eine theoretische, eine konstruktive (Paradigma der „Design Science“) oder eine empirische Arbeit (Paradigma der „Behavioral Science“).

\minisec{Schluss}\index{Schluss}

Im Schlussteil soll deutlich werden, was innerhalb der Arbeit erreicht wurde.
Dazu kann man sich auf die Anfangs beschriebenen Ziele oder Hypothesen beziehen und die wichtigsten Schritte und Erkenntnisse zusammenfassen.
Nach den Ergebnissen kann zusätzlich ein Ausblick gegeben werden, wie sich das Problem weiterentwickeln wird, oder welche weiteren Forschungsfragen sich an die Arbeit anknüpfen.

\minisec{Erklärung zur Urheberschaft}

Eine unterschriebene \emph{Erklärung zur Urheberschaft} ist der Bachelor- und der Masterarbeit beizulegen. Bei der Abgabe von digitalen Arbeiten, z.B. Hochladen einer PDF-Datei auf Grips, reicht die Erklärung ohne Unterschrift aus.
Den konkreten Text der Erklärung finden Sie in der nach dem Anhang.

\minisec{Index}\index{Index}

Bei umfangreichen Arbeiten kann ein Sachwortverzeichnis oder Index sinnvoll sein.
Der Index enthält am Ende der Arbeit bedeutende Schlagworte und dazugehörige Seitenangaben.

\minisec{Anhang}

Material, das innerhalb der Arbeit benutzt wurde oder entstanden ist wie Fragebögen, Nutzungsszenarien, umfangreiche Tabellen, statistisches Datenmaterial, Wireframes oder interaktive Prototypen wird als Anhang ganz zu Schluss der Arbeit beigefügt.
Bei mehreren Anhängen werden diese alphabetisch gekennzeichnet, z. B. „Anhang A Fragebogen“, „Anhang B Nutzungsszenarien“ usw.
Die Anhänge sind im Inhaltsverzeichnis mit aufgeführt aber nicht nummeriert.

Überschaubares Material wie Tabellen, kleinere Mockups, Fragebögen können direkt auf eine Dokumentseite platziert werden.
Umfangreiches Datenmaterial oder interaktive Inhalte können auf einer CD beigefügt werden.
Diese ist dann auf der entsprechenden Seite zu befestigen. 

\section{Formatierung}\label{subsec:formatierung}

\subsection{Seitengestaltung\index{Seitengestaltung} und Druck}\label{subsubsec:seitengestaltung}\index{Druck}

Die Arbeit wird einseitig auf DIN A4 gedruckt. Der Seitenrand beträgt oben 2,5 cm, unten 2,5 cm, links 3, 7 cm, rechts 3,5 cm. Die Kopfzeile enthält die jeweilige Kapitelüberschrift (also die Überschrift erster Ordnung).
Jede Seite, ausgenommen das Deckblatt und das Inhaltsverzeichnis enthält eine Seitennummer mittig in der Fußzeile.
Alle Arbeiten sind in angemessener Weise abhängig vom Umfang zu binden: Für Seminararbeiten genügt ein Schnellhefter oder eine Ringbindung. Für Bachelor- und Masterarbeiten sollte eine Klebe- oder Klemmbindung verwendet werden.

\subsection{Typographie und Textsatz}\label{subsubsec:typographie}

Unterschiede in der Lesbarkeit bestimmter Schrifttypen sind nicht empirisch belegt sondern eher als Faustregeln, die in der typografischen Praxis entstanden sind, zu verstehen:  Serifenschriften wie „Times New Roman“, „Garamond“, „Palatino“ oder „TeX Gyre Pagella“ (wie hier Text)  mit starken Unterschieden in der Strichstärke werden wegen ihrer guten Lesbarkeit häufig bei Texten in Büchern oder Zeitschriften eingesetzt \cite[S.18]{gotz2004typo}.
Serifenlose Schriften wie „Frutiger Next“, „Verdana“ oder „Helvetica“ mit geringen oder keinen Unterschieden in der Strichstärke werden am häufigsten in Titel oder Überschriften verwendet \cite[S.18]{gotz2004typo}. 

  Je nach Schriftart sollte die Schriftgröße für den Fließtext\index{Fließtext} 11- 12pt betragen. Der Zeilenabstand sollte 1,5 fach gewählt werden.

Für die Überschriften \index{Überschriften}kann die gleiche Schrift wie für den Fließtext gewählt werden (wie in der Vorlage) oder man verwendet eine serifenlose Schrift, da sie sich gut von der Serifenschrift im Text abgrenzt.
Beispiele sind „Arial“, die Hausschrift der Uni „Frutiger Next“ uvm.
Bei der Kombination von Schriftarten sollte man grundsätzlich vorsichtig vorgehen und sich an gelungenen Beispielen oder Empfehlungen von Typographen orientieren. 

Für Fußnoten\index{Fußnoten} sollte die gleiche Schrift wie im Fließtext verwendet werden. Die Schriftgröße sollte erkennbar kleiner sein z. B. 10 pt\footnote{Dies ist eine Fußnote}.
Für den Fließtext ist Blocksatz mit automatischer Silbentrennung zu verwenden.
Überschriften aller Art und Verzeichnisse (Literaturverzeichnis etc.) werden linksbündig ausgerichtet.
Zur besseren Lesbarkeit sollte die erste Zeile eines Absatzes eingerückt sein (hier um 0,7cm).
Für eine optimale Seitengestaltung sollte eine vereinzelte Zeile am Anfang einer Seite oder am Ende der Seite (sogenannte „Hurenkinder“ und „Schusterjungen“) vermieden werden.

Zur Hervorhebung von Begriffen können Kursivsetzung oder „Anführungszeichen“ verwendet werden.
Generell sollte man sich für eine Art der Hervorhebung entscheiden und diese dann sparsam und  konsistent verwenden. 

\subsection{Abbildungen}\label{subsubsec:abbildungen}

Abbildungen\index{Abbildungen} werden mit einem Titel und einer Quellenangabe beschriftet.
Ist die Abbildung selbst erstellt oder eine eigene Fotografie so wird dies durch „eigene Abbildung“ bzw. „eigenes Bild“ gekennzeichnet, ein Beispiel ist in Abbildung \ref{fig:abbildung} gezeigt.

\begin{figure}
	\centering
	\includegraphics[width=0.5\textwidth]{images/blume}
	\caption{Blumen (Quelle, Jahr, Seitenzahl)}
	\label{fig:abbildung}
\end{figure}

\subsection{Tabellen}\label{subsubsec:tabellen}\index{Tabellen}

{\renewcommand{\arraystretch}{1.5}
\begin{table}[h!]
\centering
\begin{tabular}{ l|l } 
\hline
\bfseries Typ & \bfseries Seitenumfang\\
\hline
Seminararbeit & 20-30 Seiten \\
Bachelorarbeit & 30-50 Seiten \\
Masterarbeit & 60-80 Seiten \\
\hline
\end{tabular}
\caption{Empfohlener Textumfang}
\label{table:textumfang}
\end{table}}

Tabellen sollten nur für Inhalte verwendet werden der auch geeignet ist für eine tabellarische Darstellung wie zum Beispiel der Vergleich von Daten.
Tabellen sind wie Abbildungen mit einer Beschriftung zu versehen.

\subsection{Code\index{Code}}\label{subsubsec:code}

Für Programmcode oder Text in Auszeichnungssprachen wie HTML sollten nichtproportionale Schriftarten wie „Courier New“, „Lucida Console“ oder „Monospace“ verwendet werden.
Die Zeichenbreite ist bei diesen Schriftarten bei jedem Zeichen gleich und die Struktur des Codes ist übersichtlicher.
Generell sollten Codeteile im Hauptteil möglichst kurz gehalten werden, bis ca. 20 Zeilen, da sonst der Lesefluss deutlich unterbrochen wird.
Damit der Programmcode besser lesbar ist und durchsucht werden kann sollte er als Text und nicht als Screenshot eingefügt werden.
Nach Möglichkeit sollte Syntax-Highlighting verwendet werden.
Für die  Erläuterung eines Algorithmus sollte der Code stark vereinfacht und kommentiert werden.


\begin{lstlisting}[language=HTML,caption={So sieht Code schön aus.},captionpos=b,label=code:arduino_blink]
<!DOCTYPE html PUBLIC "-//W3C//DTD XHTML 1.0 Strict//EN"
  "http://www.w3.org/TR/xhtml1/DTD/xhtml1-strict.dtd">
<html xmlns="http://www.w3.org/1999/xhtml" xml:lang="de" lang="de">

<head>
</head>
<body>

</body>
</html>
\end{lstlisting}


\section{Zitierweise}\label{subsec:zitierweise}

Fremde Inhalte, Ideen, Tabellen oder Abbildungen sind über Referenzen im APA-Stil nachzuweisen.
Jede Art von Verweis, direktes oder indirektes Zitat oder der Nachweis von Abbildungen besteht dabei aus einem kurzen Verweis im Text bzw. der Bildbeschriftung und einer vollständigen Auflistung aller Angaben im Literaturverzeichnis.

\subsection{Direkte Zitate}\label{subsubsec:direkte}\index{Direkte Zitate}

Direkte Zitate sind wortgenaue Übernahmen, das heißt, dass jedes Wort so lauten muss wie in der Originalquelle.
Auch grammatikalische oder orthographische Fehler sind zu übernehmen, können aber durch ein nachfolgendes [sic] gekennzeichnet werden.
Kurze Zitate werden in den Fließtext integriert und in „“ gesetzt.
Auslassungen werden mit eckigen Klammern […] sichtbar gemacht.
Auslassungen am Satzanfang werden nicht sichtbar gemacht.
Ein Beispiel: Usability wird definiert als „das Ausmaß, in dem ein Produkt durch bestimmte Nutzer in einem bestimmten Nutzungskontext genutzt werden kann, um bestimmte Ziele effektiv, effizient und zufriedenstellend zu erreichen.“ (DIN EN ISO 9241-11 1997, 94).
Längere Zitate über mehrere Zeilen sollten eingerückt, engerzeilig gesetzt und mit Anführungszeichen dargestellt werden.
Beispiel:

\begin{quote}
«Usability is usually considered the ability of the user to use the thing to carry out a task successfully, whereas user experience takes a broader view, looking at the individual’s entire interaction with the thing, as well as the thoughts, feelings and perception that results from that interaction.»
\cite[S. 4]{albert2013measuring}
\end{quote}

Direkte Zitate sollten sparsam eingesetzt werden, dort wo eine eigene Umschreibung zu einem Bedeutungsverlust oder einer Bedeutungsänderung führt.
Sinnvoll sind direkte Zitate vor allem bei Begriffsdefinitionen oder prägnanten Problemformulierungen.
Beim Zitieren aus fremdsprachigen Originalquellen sollte man keine fremdsprachigen Satzteile  in den deutschen Satz integrieren.
Hier sollte man lieber einen kompletten Satz direkt zitieren oder den benötigten Teil paraphrasieren.

\subsection{Indirekte Zitate}\label{subsubsec:indirekte}\index{Indirekte Zitate}

Indirekte Zitate sind sinngemäße Übernahmen, dass heißt sinngemäße Paraphrasen von Inhalten.
Nach der Umschreibung in einem oder mehreren Sätze folgt die Angabe der Quelle \cite[S. 11]{mustermann2013test}.
Indirekte Zitate dienen dem Beleg von Behauptungen, Methoden oder Fragestellungen in komprimierter Form.

\subsection{Sekundäre Zitate}\label{subsubsec:sekundäre}\index{Sekundäre Zitate}

Wenn der zitierte Inhalt selbst schon ein Zitat ist, handelt es sich um ein sekundäres Zitat.
Im Idealfall sollte ein sekundäres Zitat vermieden und die Originalquelle recherchiert und diese zitiert werden.
Ist die Originalquelle nicht verfügbar, so wird wie folgt referenziert: Zuerst wird die Originalquelle angegeben, danach folgt mit dem Zusatz „zitiert nach“ die Quelle, aus welcher der Inhalt übernommen wurde (\citealp[S. 11]{mustermann2013test}, zitiert nach \citealp[S. 11]{huber2013buch}).

\subsection{Zitierweise im Text}\label{subsubsec:zitierweise}\index{Zitierweise im Text}

Im APA-Stil folgt die Quellenangabe im Text nach dem direkten (Block-) Zitat oder der Paraphrase.
Grundsätzlich sollte die Angabe sobald wie möglich nach der übernommenen Aussage stehen, das heißt nicht erst nach einem ganzen Absatz.
Nachname, Erscheinungsjahr und Seitenangabe werden in runde Klammern gesetzt (Autor, Jahr, Seitenzahl).
Der Zusatz von „Vgl.“ o.ä. ist bei indirekten Zitaten nicht nötig.
Sind keine Seitenangaben verfügbar, so sollten Kapitelüberschriften oder Paragraphennummern verwendet werden.
Wird ein Ansatz oder eine Methode aus einem Aufsatz als Ganzes referenziert, müssen keine Seitenzahlen stehen \cite{mustermann2013test}.
Besitzt die Quellenangabe zwei Autoren, so werden beide genannt \cite{mustermann2015gemeinsam}.
Bei drei, vier oder fünf Autoren werden in der ersten Referenz alle genannt \cite*{mustermann2017viele} und bei weiteren Referenzen nur der erste Autor genannt und die anderen mit et al. abgekürzt \cite{mustermann2017viele}.
Die Autorennamen können auch in den Text integriert werden, dann wird nur die Jahreszahl in Klammern angeführt: \citet{mustermann2013test} vergleichen in ihrer Untersuchung.
Bei Wikipedia-Artikeln wird der Titel des Artikels oder die ersten 3-4 Wörter davon in Anführungszeichen angeführt (vor dem letzten Anführungszeichen steht ein Komma \cite{noauthor_wissenschaft_2020}.

\subsection{Angaben im Literaturverzeichnis}\label{subsubsec:angaben}\index{Literaturverzeichnis}

Das Literaturverzeichnis listet alle in der Arbeit referenzierten Quellen in alphabetischer Reihenfolge auf.
Es enthält alle Angaben, die zur Identifikation einer Quelle nötig sind.
Die Vorgaben sind je nach Publikationstyp unterschiedlich.
Die wichtigsten Typen sind unten aufgeführt.
Bei Unklarheiten oder Sonderformen kann im Handbuch der APA nachgeschlagen werden \cite{american2010concise}.

\minisec{Monographie}

% this is needed for bibliography items within the document to work, so nothing to worry about if you don't want to do this
\makeatletter 
\renewcommand\BR@b@bibitem[2][]{\BR@bibitem[#1]{#2}\BR@c@bibitem{#2}}           
\makeatother

\nobibliography*%{literature}
% -------------------------------

\bibentry{sampleMonography}

\bibentry{bortz2007forschungsmethoden}


\minisec{Herausgeberschrift}

\bibentry{sampleCollection}

\bibentry{lewandowski2009handbuch}


\minisec{Artikel oder Kapitel in einer Herausgeberschrift}

\bibentry{sampleInCollection}

\bibentry{butler2002community}


\minisec{Zeitschriftenartikel}

\bibentry{sampleJournalArticle}

\bibentry{schneiderman2002understanding}


\minisec{Beitrag in Konferenzband}

\bibentry{sampleInProceedings}

\bibentry{ehrlich2010microblogging}


\minisec{Elektronische Quellen}

\bibentry{sampleWebsite}

\bibentry{shirky2014ontology}


\minisec{Wikipedia-Artikel}

\bibentry{sampleWikipedia}

%\chapter{Empfehlungen für empirische Arbeiten}\label{sec:empfehlungen}

\section{Handbücher}\label{subsec:handbücher}

\cite{lazar2017research} bieten einen Überblick über Forschungsmethoden im Bereich der HCI. Neben Grundlagen zu experimentellem Design werden Statistische Analyse, Umfragen, Tagebücher, Fallstudien, Fokusgruppen etc. in einzelnen Kapiteln vorgestellt.

Das Handbuch von \cite{sauro2016quantifying} ist ein praktischer Ratgeber zum Einsatz statistischer Methoden im Bereich des Usability-Testing. Er zeigt auf, was in einem Usabilty- Test erhoben werden kann, und welche Tests zur Auswertung herangezogen werden können. Auch die Frage nach der geeigneten Stichprobengröße wird ausführlich unter Berücksichtigung des jeweiligen Untersuchungsziels beantwortet.

Das Handbuch von \cite{rubin2008handbook} bietet eine Schritt-für-Schritt Anleitung für die Organisation, Durchführung und Dokumentation eines Usability-Tests.


\section{Darstellung der Ergebnisse}\label{subsec:darstellung}

Die Ergebnisse einer empirischen Studie sind nachvollziehbar und in einem angemessenen Detailgrad darzustellen. In einer rein empirischen Arbeit sollten die Ergebnisse mit größerer Ausführlichkeit betrachtet werden, als in einer Entwicklungsarbeit, in der die Evaluation nur einen Teilbereich darstellt. Zu vermeiden ist das bloße Aufzählen von Einzelergebnissen oder Beobachtungen. Eine sinnvolle Gliederung für die meisten Ergebnisse ist die Folgende:

\begin{enumerate}
    \item{Überblick oder Zusammenfassung der wichtigsten Ergebnisse}
    \item{Methode }
    \item{Einzelne Befunde }
    \item{Empfehlungen für das Redesign}
\end{enumerate}

Je nach Ausführlichkeit und Zahl der Beobachtungen kann der Abschnitt Befunde nach einzelnen Problemen (als Überschriften) oder nach Aufgaben bzw. Szenarien in denen die Probleme auftraten organsisiert werden. Anregungen für die Gestaltung von Usability Reports bieten auch viele online verfügbare Templates (i.e. \cite{department_usability_2013}, Abschnitt Templates)

%\chapter{Zusammenfassung}\label{sec:zusammenfassung}

Die hier vorliegende Dokumentvorlage soll die Gestaltung wissenschaftlicher Arbeiten am Lehrstuhl für Medieninformatik erleichtern und zur Qualitätssicherung beitragen. Es werden Richtlinien für den inhaltlichen Aufbau und die formale Gestaltung formuliert. Dabei ist die Vorlage selbst wie eine wissenschaftliche Arbeit strukturiert und enthält die wichtigsten Formatvorlagen zur effizienten Gestaltung mit Word. Als zusätzliches Hilfsmittel für den inhaltlichen Aufbau verschiedener Arbeiten befinden sich im Anhang typische Bausteine bzw. Mustergliederungen für verschiedene Thementypen, wie „theoretische“, „konstruktive“ oder „empirische“ Arbeiten. Für weitere Informationen sind im Kapitel \ref{sec:stand_der_technik} verschiedene Ratgeber kurz vorgestellt. 


% Kapitelbezeichnung in der rechten oberen Ecke entfernen
\clearpage
\pagestyle{plain}

% Literaturverzeichnis anzeigen
% kleinerer Zeilenabstand, damit es nicht so gestreckt aussieht
\onehalfspacing
\bibliography{literature}
\onehalfspacing

% Anhang
\appendix
\chapter{Docker Compose des Metasploit Containers}\label{sec:anhang}\index{Demo}

\begin{minted}[linenos]{yaml}
version: "3.9"

services:
  metasploit:
    build: "MetasploitContainer"
    depends_on:
      - postgres
      - redis
    sysctls:
      - net.ipv6.conf.all.disable_ipv6=0
      - net.ipv6.conf.all.accept_ra=2
      - net.ipv6.conf.all.forwarding=1
    cap_add:
      - NET_ADMIN         # Allows us to create 
                          # a tap interface in the container
      - NET_RAW           # Required for iptables
      - NET_BIND_SERVICE  # Bind a socket to internet domain
                          # privileged ports 
                          # (port numbers less than 1024).
      - SETPCAP           # Modify process capabilities.
    environment:
      - TAPSERVERIP=192.168.1.128/24 # Sets the IP of the tapserver
    ports:
      - "2222:22"         # SSH Port
      - "33000:33000/udp" # Tapserver ServiceDiscoveryPort
      - "37471:37471"     # Tapserver DataPort
      - "55553:55553"     # MsfRPC
    volumes:
      - ./modules/:/root/.msf4/modules
#      - ./modulesDev/:/root/dev/

  redis:
    image: redislabs/rejson
    ports:
      - "6379:6379"
\end{minted}

\chapter{Screenshot PentestStudio}\label{sec:penteststudio}\index{Another}

\includegraphics[height=\textheight - 4cm,keepaspectratio]{images/anhang2.png}

\chapter{Evaluationsaufbau}\label{sec:evaluationsaufbau}\index{anotheraufbau}

\includegraphics[angle=90, height=\textheight - 4cm,keepaspectratio]{images/idaufbau.jpeg}


% Tipps zur Verwendung von LaTeX
%\chapter{Tipps für \LaTeX}\label{latex}

Hier finden Sie gesammelte Hinweise, die sich speziell auf das \LaTeX-Template und dessen Verwendung beziehen. Der Aufbau des Templates wird dargelegt und wichtige Befehle werden anhand von Beispielen erklärt.

\section{Struktur des Templates}

\begin{description}
    \item[mi-document/]{Dieses Verzeichnis enthält die Style- und Class-Files, die das eigentliche Template darstellen. In diesen werden Pakete importiert, Formatierungen definiert und zusätzliche Funktionen bereitgestellt. Manche Entwicklungsumgebungen für \LaTeX, wie TeXmaker und TeXstudio, finden diese Dateien leider nur, wenn sie sich im gleichen Verzeichnis befinden wie \verb|document.tex|. In diesem Fall müssen alle \verb|.sty|- und \verb|.cls|-Dateien in dessen Verzeichnis verschoben werden.}
    \item[document.tex]{Diese Datei enthält den Quellcode für dieses Dokument. Es wird empfohlen, Arbeiten auf diese Datei aufzubauen. In den ersten Zeilen werden Variablen definiert (z.B. Autor, Titel der Arbeit, etc.), welche vom Template beispielsweise für das Deckblatt und Metadaten verwendet werden. Danach werden Inhalts-, Abbildungs-, Tabellen- und Codeverzeichnis angezeigt. Der tatsächliche Inhalt der Arbeit kann direkt in diese Datei geschrieben werden, es wird jedoch empfohlen, Kapitel auf verschiedene Dateien aufzuteilen und über den \verb|\input|-Befehl einzubinden.}
    \item[lizenzierung.tex]{Hier werden Lizenzen für die schriftliche Ausarbeitung und den Code der Arbeit sowie ein eventueller Sperrvermerk (nur nach Absprache!) angegeben. Passen Sie diese Datei so an, dass sie auf Ihre Arbeit zutrifft. Mit den Befehlen \verb|\checkboxEmpty| und \verb|\checkboxChecked| können leere und angekreuzte Kästchen erstellt werden.}
    \item[hinweise.tex]{Diese Datei enthält die Präambel dieser Formatvorlage und sollte \textbf{nicht} in Ihrer fertigen Arbeit enthalten sein. Entfernen Sie dazu den Import dieser Datei aus \verb|document.tex|.}
    \item[*.tex]{Die anderen \verb|.tex|-Dateien enthalten den Inhalt dieses Dokuments (eine Datei pro Kapitel). Sie werden von \verb|document.tex| importiert.}
    \item[images/]{Bilddateien in diesem Verzeichnis können als Abbildungen in ein \LaTeX-Dokument eingebunden werden.}
\end{description}

\section{Wichtige Befehle}

\subsection{Sections, Subsections, etc.}

Die Abschnitte eines \LaTeX-Dokuments werden in \emph{chapters}, \emph{sections} und \emph{subsections} aufgeteilt.
Diese Abschnitte werden automatisch mit einer fortlaufenden Nummerierung versehen und im Inhaltsverzeichnis aufgeführt.
Die Syntax der Befehle lautet wie folgt:

\begin{lstlisting}[language=LaTeX]
\chapter{Kapitelname}              % z.B. 4 Gestaltungsrichtlinien
\section{Unterkapitelname}         % z.B. 4.3 Formatierung
\subsection{Unterunterkapitelname} % z.B. 4.3.3 Abbildungen
\end{lstlisting}

Es wir empfohlen, Abschnitte außerdem mit Labels zu versehen, damit sie im späteren Verlauf der Arbeit einfacher referenziert werden können:

\begin{lstlisting}[language=LaTeX]
\chapter{Kapitelname}\label{sec:Kapitelname} % Label wird gesetzt
\ref{sec:Kapitelname}                        % Referenz auf das Label
\end{lstlisting}

Das die Referenz wird im Fließtext durch die \textbf{Kapitelnummer} ersetzt, sodass eine Syntax wie \verb|(siehe Kapitel \ref{Kapitelname})| empfohlen wird.

Ist ein Abschnitt \textbf{ohne} Nummerierung gewünscht, so können die Befehle \verb|\addchap| und \verb|\addsec| genutzt werden. Diese erzeugen nicht-nummerierte Überschriften inklusive Eintrag ins Inhaltsverzeichnis.

Darüber hinaus existieren noch kleinere Zwischen-Überschriften, die nicht in das Inhaltsverzeichnis eingetragen werden.

\begin{lstlisting}[language=LaTeX]
\minisec{Überschrift}
\end{lstlisting}



\subsection{Listen}

\LaTeX{} stellt drei verschiedene Arten von Listen bereit: Aufzählungen \emph{(itemize)}, nummerierte Aufzählungen \emph{enumerate} und Beschreibungen \emph{description}.
Beispielhaft werden hier Quellcode und Resultat dargestellt:

\minisec{Aufzählung}

\begin{lstlisting}[language=LaTeX]
\begin{itemize}
    \item{Das ist der erste Punkt.}
    \item{Das ist der zweite Punkt.}
    \item{Das ist der dritte Punkt.}
\end{itemize}
\end{lstlisting}

\begin{itemize}
    \item{Das ist der erste Punkt.}
    \item{Das ist der zweite Punkt.}
    \item{Das ist der dritte Punkt.}
\end{itemize}

\minisec{Nummerierte Aufzählung}

\begin{lstlisting}[language=LaTeX]
\begin{enumerate}
    \item{Das ist der erste Punkt.}
    \item{Das ist der zweite Punkt.}
    \item{Das ist der dritte Punkt.}
\end{enumerate}
\end{lstlisting}

\begin{enumerate}
    \item{Das ist der erste Punkt.}
    \item{Das ist der zweite Punkt.}
    \item{Das ist der dritte Punkt.}
\end{enumerate}

\minisec{Beschreibung}

\begin{lstlisting}[language=LaTeX]
\begin{description}
    \item[Erstens]{Das ist der erste Punkt.}
    \item[Zweitens]{Das ist der zweite Punkt.}
    \item[Drittens]{Das ist der dritte Punkt.}
\end{description}
\end{lstlisting}

\begin{description}
    \item[Erstens]{Das ist der erste Punkt.}
    \item[Zweitens]{Das ist der zweite Punkt.}
    \item[Drittens]{Das ist der dritte Punkt.}
\end{description}

\noindent Unterpunkte können erstellt werden, indem innerhalb einer Liste eine neue Liste erstellt wird:

\begin{lstlisting}[language=LaTeX]
\begin{itemize}
    \item{Das ist der erste Punkt.}
    \item{Das ist der zweite Punkt.}
    \begin{itemize}
        \item{Das ist ein Unterpunkt.}
        \item{Das ist noch ein Unterpunkt.}
    \end{itemize}
\end{itemize}
\end{lstlisting}

\begin{itemize}
    \item{Das ist der erste Punkt.}
    \item{Das ist der zweite Punkt.}
    \begin{itemize}
        \item{Das ist ein Unterpunkt.}
        \item{Das ist noch ein Unterpunkt.}
    \end{itemize}
\end{itemize}

\subsection{Abbildungen, Tabellen und Code}

Größere Objekte, wie Tabellen oder Bilder, sollten in der Regel nicht an einer festen Position platziert werden.
Der Seitenumbruch kann sonst nicht optimal platziert werden und bei der Platzierung mitten auf der Seite, wird der Lesefluss unterbrochen.
\LaTeX{} liefert hier dem Autor die Möglichkeit sogenannte \emph{Gleitobjekte} mithilfe einer \emph{Gleitumgebung} zu definieren.

Dies führt dazu, dass Objekte möglichst sinnvoll platziert werden, jedoch nicht direkt an der Stelle, an der sie definiert wurden, vgl. Abbildung \ref{fig:beispiel}.
Die nicht-exakte Positionierung des Objektes ist dabei kein Nachteil, da ohnehin jede Abbildung über eine entsprechende Beschriftung und Nummer verfügen muss.
Dadurch ist die Zuordnung mithilfe von Querverweisen möglich.

Die Gleitumgebung heißt für Abbildungen \emph{figure} und für Tabellen \emph{table}. Die folgenden Mechanismen sind dabei analog benutzbar, es ändert sich lediglich der Bezeichner an der Beschriftung.

\begin{lstlisting}[language=LaTeX]
\begin{figure}
<hier wird das eigentliche Bild eingefügt, siehe Folgeabschnitt>
\caption{Bildbeschreibung}
\label{fig:label}
\end{figure}
\end{lstlisting}


Der Querverweis auf die Abbildung ist dann folgendermaßen möglich:
\begin{lstlisting}[language=LaTeX]
siehe Abbildung \ref{fig:beispiel}
\end{lstlisting}
es erzeugt die Ausgabe: „siehe Abbildung \ref{fig:beispiel}“

Alle Gleitumgebungen verfügt zusätzlich über ein Optionales Argument, dass zur Beeinflussung der Platzierung benutzt werden kann:
\begin{lstlisting}[language=LaTeX]
\begin{figure}[Platzierung „tbp“, falls nicht angegeben]
\end{lstlisting}

Dieses Argument kann eine Prioritätenliste aus t „top“, b „bottom“, h „here“ oder p „page“ (extra
Seite) gesetzt werden. Der erste Wert der Liste hat die höchste Priorität und wird wenn möglich verwirklicht.
Voreingestellt ist tbp. Dies ist dadurch begründet, dass h eine Positionierung mitten auf der Seite
bedeutet. Die daraus resultierende Spaltung des Satzspiegels ist typografisch fragwürdig und wird daher nicht empfohlen.


\minisec{Abbildungen}

Um eine Abbildung einzufügen kann obiges Beispiel entsprechend erweitert werden:

\begin{lstlisting}[language=LaTeX]
\begin{figure}
    \centering
    \includegraphics[width=0.5\textwidth]{images/dateiname}
    \caption{Bildbeschreibung}
    \label{fig:label}
\end{figure}
\end{lstlisting}

\begin{figure}
    \centering
    \includegraphics[width=0.5\textwidth]{images/blume}
    \caption{Bildbeschreibung des Beispielbildes}
    \label{fig:beispiel}
\end{figure}

Meist sind ausführliche Bildbeschreibungen empfehlenswert, damit eine Abbildung für sich selbst stehen kann.
In diesem Fall kann durch einen optionalen Parameter \verb|\caption[Kurzbeschreibung]{Langbeschreibung}| eine Kurzform angegeben werden, welche dann im Abbildungsverzeichnis erscheint.

\begin{wrapfigure}{R}{0.3\textwidth}
    \centering
    \includegraphics[width=0.25\textwidth]{images/schwammerl}
    \caption{Bild mit Textumfluss.}
    \label{fig:beispiel_umfluss}
\end{wrapfigure}

Bei sehr hohen und schmalen Bildern kann es nützlich sein, Textumfluss zu verwenden.
Dies kann in \LaTeX mit der \verb|wrapfigure|-Umgebung umgesetzt werden:

\begin{lstlisting}[language=LaTeX]
\begin{wrapfigure}{R}{0.3\textwidth}
    \centering
    \includegraphics[width=0.25\textwidth]{images/schwammerl}
    \caption{Bild mit Textumfluss.}
    \label{fig:beispiel_umfluss}
\end{wrapfigure}
\end{lstlisting}



Falls Ihre Arbeit viele Abbildungen des gleichen Formates enthält, können Sie sich folgendermaßen ein Makro als Abkürzung definieren:

\begin{lstlisting}[language=LaTeX]
\newcommand*{\image}[2]{
    \begin{figure}
    \centering
    \includegraphics[width=0.5\textwidth]{images/#1}
    \caption{#2}
    \label{fig:#1}
    \end{figure}
}
\end{lstlisting}
Diese Definition steht in der Datei \emph{config.tex} und kann dort auch angepasst werden.

Damit erhalten Sie die Möglichkeit statt der obigen Umgebung einfach
\begin{verbatim}
\image{dateiname}{Bildbeschreibung}
\end{verbatim}
zu benutzen. Das Label erhält in diesem Fall den Namen „fig:<Dateiname>” für das Beispielbild.

\image{blume}{zweite Verwendung des Beispielbildes mit dem image-Makro. Die Datei heißt blume.jpg.}

Die Referenz \ref{fig:blume} wird dann mit \verb+\ref{fig:blume}+ erzeugt.

\minisec{Tabellen}

Tabellen in \LaTeX{} zu erstellen mag anfangs etwas umständlich erscheinen, sie bieten jedoch sehr viel gestalterische Möglichkeiten. Hier werden lediglich die einfachsten Grundlagen gezeigt.

Für die Platzierung kann, wie zu Beginn dieses Abschnittes Beschrieben die \emph{table}-Umgebung benutzt werden.

Innerhalb dieser befindet sich mit dem \emph{tabular}-Environment die eigentliche Tabelle.
Dessen Parameter bestimmen die Anzahl der Spalten, sowie die Ausrichtung der Inhalte dieser.
Über \verb|\hline| können horizontale Linien eingebaut werden.
Die Inhalte werden Zeilenweise (getrennt durch \verb|\\|) hinzugefügt und diese Zeilen werden durch das \verb|&|-Symbol in Spalten unterteilt.

Die Anpassung des Makros \verb+\arraystretch+ erhöht den Zeilenabstand. Für elegantere Lösungen betreffend Tabellen sei auf Ergänzungspakete wie tabularx \citep{latex:tabularx} oder booktabs \citep{latex:booktabs} verwiesen.

\begin{lstlisting}[language=LaTeX]
\begin{table}
\renewcommand{\arraystretch}{1.5}
\centering
\begin{tabular}{l|l}
\hline
\bfseries Erste Spalte& \bfseries Zweite Spalte \\
\hline
Erste Zeile & lorem \\
Zweite Zeile & ipsum \\
Dritte Zeile & dolor \\
\hline
\end{tabular}
\caption{Tabellenbeschreibung}
\label{table:label_der_tabelle}
\end{table}
\end{lstlisting}

\begin{table}
    \renewcommand{\arraystretch}{1.5}
    \centering
    \begin{tabular}{l|l}
        \hline
        \bfseries Erste Spalte& \bfseries Zweite Spalte \\
        \hline
        Erste Zeile & lorem \\
        Zweite Zeile & ipsum \\
        Dritte Zeile & dolor \\
        \hline
    \end{tabular}
    \caption{Tabellenbeschreibung}
    \label{table:label_der_tabelle}
\end{table}

\minisec{Code}

Um Code einzubinden, wird das \verb|lstlisting|-Environment verwendet.
Über Parameter können Metadaten wie die Programmiersparche (wichtig für korrektes Syntax-Highlighting), die Beschreibung und das Label festgelegt werden.

\begin{verbatim}
\begin{lstlisting}[language=C++,
                   caption={Beschreibung des Codebeispiels},
                   captionpos=b,
                   label=code:code_label]
#include <stdio.h>

int main()
{
    printf("Hallo Welt!\n");
}
\end{lstlisting}
\end{verbatim}

\begin{lstlisting}[language=C++,
                   caption={Beschreibung des Codebeispiels},
                   captionpos=b,
                   label=code:code_label]
#include <stdio.h>

int main()
{
    printf("Hallo Welt!\n");
}
\end{lstlisting}

Über das \verb|verbatim|-Environment kann mehrzeiliger Code \textbf{ohne} Syntax-Highlighting, Beschreibung und Label eingebunden werden.

\verb|\begin{verbatim}|

\verb|Dieser Text ist in einer Monosaced-Schriftart geschrieben.|

\verb|Könnte z.B. für Pseudocode nützlich sein.|

\verb|\end{verbatim}|

\begin{verbatim}
Dieser Text ist in einer Monosaced-Schriftart geschrieben.
Könnte z.B. für Pseudocode nützlich sein.
\end{verbatim}

Sollen nur einzelne Wörter monospaced angezeigt werden, kann auf \verb!\verb|text|! zurückgegriffen werden: \verb|text|.

\subsubsection{Zitate und Fußnoten}

\minisec{Zitate im Text}

Zitate im Text und Einträge im Literaturverzeichnis werden aus Einträgen einer \emph{BibTeX}-Datei generiert.
Diese kann entweder selbst zusammengestellt, oder aus einem Programm zu Literatuverwaltung (z.B. Zotero, Mendeley, Citavi) exportiert werden.

Um eine Quelle zu zitieren, können folgende Befehle verwendet werden:

\bigskip

\begin{tabular}{@{}ll@{}}
\hline
\textbf{Befehl} & \textbf{Ergebnis} \\
\hline
    \verb|\cite{mustermann2013test}| & \cite{mustermann2013test} \\
    \verb|\cite[S. 15]{mustermann2013test}| & \cite[S. 15]{mustermann2013test} \\
    \verb|\citet[S. 15]{mustermann2013test}| & \citet[S. 15]{mustermann2013test} \\
\hline
\end{tabular}

\bigskip

Einträge im Literaturverzeichnis werden automatisch erstellt, wenn eine Quelle zitiert wird.

\minisec{Blockzitate}

Blockzitate werden mit dem \verb|quote|-Environment umschlossen.
Sie sollten mit einem Verweis auf die Literaturquelle beendet werden.

\begin{lstlisting}[language=LaTeX]
\begin{quote}
«Wenn die Wurst so[…]»
\cite[S. 4]{mustermann2013test}
\end{quote}
\end{lstlisting}

\begin{quote}
«Wenn die Wurst so dick wie das Brot ist, ist es wurst wie dick das Brot ist.»
\cite[S. 4]{mustermann2013test}
\end{quote}

\minisec{Fußnoten}

Fußnoten werden über den \verb|\footnote{Inhalt der Fußnote}|-Befehl erstellt und automatisch nummeriert. Beispiel:

Informationen zum Studium der Medieninformatik an der Universität Regensburg finden Sie auf der Website des Lehrstuhls\footnote{\url{https://www.uni-regensburg.de/sprache-literatur-kultur/medieninformatik/}}.


\begin{singlespace}
\KOMAoptions{parskip=full}
% Erklärung zur Urherberschaft (urheberschaft.tex) anhängen
\addchap{Erklärung zur Urheberschaft}

Hiermit erkläre ich eidesstattlich, dass die vorliegende Arbeit von mir selbstständig und ohne unerlaubte Hilfe angefertigt wurde, insbesondere, dass ich alle Stellen, die wörtlich oder annähernd wörtlich oder dem Gedanken nach aus Veröffentlichungen und unveröffentlichten Unterlagen und Gesprächen entnommen worden sind, als solche an den entsprechenden Stellen innerhalb der Arbeit durch Zitate kenntlich gemacht habe, wobei in den Zitaten jeweils der Umfang der entnommenen Originalzitate kenntlich gemacht wurde. Ich bin mir bewusst, dass eine falsche Versicherung rechtliche Folgen haben wird.

Die vorgelegten Druckexemplare und die vorgelegte digitale Version sind identisch.

\ifdefstring
{\getWorkType}{Masterarbeit}
{Von den zu § 27 Abs. 5 der Prüfungsordnung vorgesehenen Rechtsfolgen habe ich Kenntnis.}
{}

\signature


% Erklärung zur Lizenzierung der Arbeit (lizenzierung.tex) anhängen
%\addchap{Erklärung zur Lizenzierung und Publikation dieser Arbeit}

\textbf{Name:} \getAuthor

\textbf{Titel der Arbeit:} \textit{\getTitle}

Hiermit gestatte ich die Verwendung der schriftlichen Ausarbeitung zeitlich unbegrenzt und nicht-exklusiv unter folgenden Bedingungen:

\begin{itemize}
    \item[\checkboxEmpty] Nur zur Bewertung dieser Arbeit
    \item[\checkboxEmpty] Nur innerhalb des Lehrstuhls im Rahmen von Forschung und Lehre
    \item[\checkboxChecked] Unter einer Creative-Commons-Lizenz mit den folgenden Einschränkungen:
    \begin{itemize}
        \item[\checkboxChecked] BY – Namensnennung des Autors
        \item[\checkboxEmpty] NC – Nichtkommerziell
        \item[\checkboxEmpty] SA – Share-Alike, d.h. alle Änderungen müssen unter die gleiche Lizenz gestellt werden.
    \end{itemize}
\end{itemize}
{\scriptsize(An Zitaten und Abbildungen aus fremden Quellen werden keine weiteren Rechte eingeräumt.)}

Außerdem gestatte ich die Verwendung des im Rahmen dieser Arbeit erstellten Quellcodes unter folgender Lizenz:

\begin{itemize}
    \item[\checkboxEmpty] Nur zur Bewertung dieser Arbeit
    \item[\checkboxEmpty] Nur innerhalb des Lehrstuhls im Rahmen von Forschung und Lehre
    \item[\checkboxEmpty] Unter der CC-0-Lizenz (= beliebige Nutzung)
    \item[\checkboxChecked] Unter der MIT-Lizenz (= Namensnennung)
    \item[\checkboxEmpty] Unter der GPLv3-Lizenz (oder neuere Versionen)
\end{itemize}

{\scriptsize(An explizit mit einer anderen Lizenz gekennzeichneten Bibliotheken und Daten werden keine weiteren Rechte eingeräumt.)}

\noindent
Ich willige ein, dass der Lehrstuhl  für Medieninformatik diese Arbeit – falls sie besonders gut ausfällt - auf dem Publikationsserver der Universität Regensburg veröffentlichen lässt.

\noindent
Ich übertrage deshalb der Universität Regensburg das Recht, die Arbeit elektronisch zu speichern und in Datennetzen öffentlich zugänglich zu machen. Ich übertrage der Universität Regensburg ferner das Recht zur Konvertierung zum Zwecke der Langzeitarchivierung unter Beachtung der Bewahrung des Inhalts (die Originalarchivierung bleibt erhalten).

\noindent
Ich erkläre außerdem, dass von mir die urheber- und lizenzrechtliche Seite (Copyright) geklärt wurde und Rechte Dritter der Publikation nicht entgegenstehen.

\begin{itemize}
    \item[\checkboxChecked] Ja, für die komplette Arbeit inklusive Anhang
    \item[\checkboxEmpty] Ja, für eine um vertrauliche Informationen gekürzte Variante (auf dem Datenträger beigefügt)
    \item[\checkboxEmpty] Nein
    \item[\checkboxEmpty] Sperrvermerk bis (Datum):
    %Sperrvermerke sind mit dem Betreuer am Lehrstuhl abzustimmen. Sperrvermerke mit einer Frist von mehr als zwei Jahren benötigen immer eine schriftliche Begründung, aus der hervorgeht, weshalb eine kürzere Sperrfrist nicht ausreichend ist.
\end{itemize}

\signature


% Stichwortverzeichnis anzeigen. Weiß nicht, warum das nicht nach dem Inhaltsverzeichnis kommt.
\end{singlespace}

% Inhalt des Datenträgers. Gehört meiner Meinung nach zum Anhang, aber was weiß ich schon. AS
%\newpage
%\input{datenträger}

\end{document}
